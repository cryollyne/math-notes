%%%%%%%%%%%%%%%%%%%%%%%%%%%%% Define Article %%%%%%%%%%%%%%%%%%%%%%%%%%%%%%%%%%
\documentclass{article}
%%%%%%%%%%%%%%%%%%%%%%%%%%%%%%%%%%%%%%%%%%%%%%%%%%%%%%%%%%%%%%%%%%%%%%%%%%%%%%%

%%%%%%%%%%%%%%%%%%%%%%%%%%%%% Using Packages %%%%%%%%%%%%%%%%%%%%%%%%%%%%%%%%%%
\usepackage{geometry}
\usepackage{graphicx}
\usepackage{amssymb}
\usepackage{amsmath}
\usepackage{amsthm}
\usepackage{empheq}
\usepackage{mdframed}
\usepackage{booktabs}
\usepackage{lipsum}
\usepackage{graphicx}
\usepackage{color}
\usepackage{psfrag}
\usepackage{pgfplots}
\usepackage{bm}
%%%%%%%%%%%%%%%%%%%%%%%%%%%%%%%%%%%%%%%%%%%%%%%%%%%%%%%%%%%%%%%%%%%%%%%%%%%%%%%

% Other Settings

%%%%%%%%%%%%%%%%%%%%%%%%%% Page Setting %%%%%%%%%%%%%%%%%%%%%%%%%%%%%%%%%%%%%%%
\geometry{a4paper}

%%%%%%%%%%%%%%%%%%%%%%%%%% Define some useful colors %%%%%%%%%%%%%%%%%%%%%%%%%%
\definecolor{ocre}{RGB}{243,102,25}
\definecolor{mygray}{RGB}{243,243,244}
\definecolor{deepGreen}{RGB}{26,111,0}
\definecolor{shallowGreen}{RGB}{235,255,255}
\definecolor{deepBlue}{RGB}{61,124,222}
\definecolor{shallowBlue}{RGB}{235,249,255}
%%%%%%%%%%%%%%%%%%%%%%%%%%%%%%%%%%%%%%%%%%%%%%%%%%%%%%%%%%%%%%%%%%%%%%%%%%%%%%%

%%%%%%%%%%%%%%%%%%%%%%%%%% Define an orangebox command %%%%%%%%%%%%%%%%%%%%%%%%
\newcommand\orangebox[1]{\fcolorbox{ocre}{mygray}{\hspace{1em}#1\hspace{1em}}}
%%%%%%%%%%%%%%%%%%%%%%%%%%%%%%%%%%%%%%%%%%%%%%%%%%%%%%%%%%%%%%%%%%%%%%%%%%%%%%%

%%%%%%%%%%%%%%%%%%%%%%%%%%%% English Environments %%%%%%%%%%%%%%%%%%%%%%%%%%%%%
\newtheoremstyle{mytheoremstyle}{3pt}{3pt}{\normalfont}{0cm}{\rmfamily\bfseries}{}{1em}{{\color{black}\thmname{#1}~\thmnumber{#2}}\thmnote{\,--\,#3}}
\newtheoremstyle{myproblemstyle}{3pt}{3pt}{\normalfont}{0cm}{\rmfamily\bfseries}{}{1em}{{\color{black}\thmname{#1}~\thmnumber{#2}}\thmnote{\,--\,#3}}
\theoremstyle{mytheoremstyle}
\newmdtheoremenv[linewidth=1pt,backgroundcolor=shallowGreen,linecolor=deepGreen,leftmargin=0pt,innerleftmargin=20pt,innerrightmargin=20pt,]{theorem}{Theorem}[section]
\theoremstyle{mytheoremstyle}
\newmdtheoremenv[linewidth=1pt,backgroundcolor=shallowBlue,linecolor=deepBlue,leftmargin=0pt,innerleftmargin=20pt,innerrightmargin=20pt,]{definition}{Definition}[section]
\theoremstyle{myproblemstyle}
\newmdtheoremenv[linecolor=black,leftmargin=0pt,innerleftmargin=10pt,innerrightmargin=10pt,]{problem}{Problem}[section]
%%%%%%%%%%%%%%%%%%%%%%%%%%%%%%%%%%%%%%%%%%%%%%%%%%%%%%%%%%%%%%%%%%%%%%%%%%%%%%%

%%%%%%%%%%%%%%%%%%%%%%%%%%%%%%% Plotting Settings %%%%%%%%%%%%%%%%%%%%%%%%%%%%%
\usepgfplotslibrary{colorbrewer}
\pgfplotsset{width=8cm,compat=1.9}
%%%%%%%%%%%%%%%%%%%%%%%%%%%%%%%%%%%%%%%%%%%%%%%%%%%%%%%%%%%%%%%%%%%%%%%%%%%%%%%

%%%%%%%%%%%%%%%%%%%%%%%%%%%%%%% Title & Author %%%%%%%%%%%%%%%%%%%%%%%%%%%%%%%%
\title{Homogeneous Equations}
\author{Patrick Chen}
\date{Sept 16, 2024}
%%%%%%%%%%%%%%%%%%%%%%%%%%%%%%%%%%%%%%%%%%%%%%%%%%%%%%%%%%%%%%%%%%%%%%%%%%%%%%%

\begin{document}
    \maketitle
    \subsection*{Ex}
    \begin{align*}
        v_1 = \begin{bmatrix}
            1 \\ 0 \\ 0
        \end{bmatrix} &&
        v_2 = \begin{bmatrix}
            1 \\ 1 \\ 0
        \end{bmatrix} &&
        v_3 = \begin{bmatrix}
            1 \\ 1 \\ 1
        \end{bmatrix} &&
    \end{align*}
    Does $v_1\dots v_3$ span $\mathbb{R}^3$?
    \begin{align*}
        A = \begin{bmatrix}
            1 & 1 & 1 \\
            0 & 1 & 1 \\
            0 & 0 & 1
        \end{bmatrix}
    \end{align*}
    Since the matrix is already in row echelon form, it can span all of
    $\mathbb{R}^3$, but it does not tell how to get any individual vector. Note
    that if you have n vectors, at most, it could span $\mathbb{R}^n$. The
    dimension is the minimum amount of vectors required to span a space.

    \section*{Linearity of matrix multiplication}
    $A$ is a matrix, $x$, $y$ are vectors, and $\lambda$ is a scalar:
    \begin{align*}
        A(x+y) &= Ax+Ay \\
        A(\lambda x) &= \lambda Ax
    \end{align*}

    \section*{Homogeneous Systems}

    A homogeneous system is a system in the form of $Ax = 0$. All homogeneous
    systems has one trivial solution: $x=0$ and may have non-trivial solutions.

    \begin{theorem}
        Any homogeneous system with more unknowns than equations has infinitely
        many solutions
    \end{theorem}

    Suppose we have a augmented matrix for a homogeneous system with more
    variables than equations with $m$ columns and $n$ rows. If $m>n$, there will
    be a free variable, and thus, infinitely many solutions.
    \begin{align*}
        \begin{bmatrix}
            A & \vec{0}
        \end{bmatrix}
    \end{align*}

    Suppose that $Ax=b$ has a solution $x_0$ and $y$ is any solution to $Ax=0$.
    $x_0 + y$ is a solution to $Ax=b$.
    \begin{align*}
        A(x_0+y) = Ax_0 + Ay = b+0 = b \\
        A(x_0-x_1) = Ax_0 - Ax_1 = b-b = 0 \\
    \end{align*}

    \subsection*{Linearly Independence}
    A set of vectors $v_1\dots v_m$ is linearly independent if in $\lambda_1 v_1
    + \dots + \lambda_m v_m = 0$, the only solution is $\lambda_1 = \dots =
    \lambda_m = 0$. This means that for $A=[v_1 \dots v_m]$, the only solution
    to $Ax=0$ is $x=0$.

    \begin{itemize}
        \item a set of any one individual vector (except for the zero vector) is
            a set of linear independent vectors in $\mathbb{R}^n$.

        \item With a set of size two, any two vectors that are not scalar
            multiples of each other are linearly independent.

        \item In a set of m vectors in $\mathbb{R}^n$, a new linearly
            independent set can be created by adding a vector that is not in the
            span of the set.

        \item  Any subset of a linearly independent set of vectors will also be
            linear dependent.
    \end{itemize}

    If $v_1\dots v_m$ are in $\mathbb{R}^n$ and $m>n$, then $v_1\dots v_m$ is not
    linearly independent.
    Since there are more columns than rows,
    the row reduction of $[v_1\dots v_m]$ will have free variables, and thus a non-trivial solution
    to $Ax=0$.

    \subsection*{Example 1}
    \begin{align*}
        v_1 = \begin{bmatrix}
            1 \\ 0 \\ 0
        \end{bmatrix} &&
        v_2 = \begin{bmatrix}
            0 \\ 1 \\ 0
        \end{bmatrix} &&
        v_3 = \begin{bmatrix}
            0 \\ 0 \\ 1
        \end{bmatrix}
    \end{align*}
    \begin{align*}
        \lambda_1 \begin{bmatrix}
            1 \\ 0 \\ 0
        \end{bmatrix} +
        \lambda_2 \begin{bmatrix}
            0 \\ 1 \\ 0
        \end{bmatrix} +
        \lambda_3 \begin{bmatrix}
            0 \\ 0 \\ 1
        \end{bmatrix} = 0
    \end{align*}
    \begin{align*}
        \begin{matrix}
            \lambda_1 &+& 0 &+& 0 &=& 0 \\
            0 &+& \lambda_2 &+& 0 &=& 0 \\
            0 &+& 0 &+& \lambda_2 &=& 0 \\
        \end{matrix}
    \end{align*}
    \begin{align*}
        \lambda_1 = 0 \\
        \lambda_2 = 0 \\
        \lambda_3 = 0
    \end{align*}

    \subsection*{Example 2}
    Is $[1, 0, 0]^T$ and $[0, 1, 0]^T$ linearly independent in $\mathbb{R}^3$?
    Yes, when it is row reduced, it does not yield any free variables. They are
    also subsets of the linearly independent set in example 1, and are thus also
    linear independent.

    \subsection*{Example 3}
    Is $v_1=[1, 2, 3]^T, v_2=[2, 1, 0]^T, v_3=[3, 3, 3]^T$ linearly independent?
    \begin{align*}
        v_1 + v_2 - v_3 = 0
    \end{align*}
    Since there is a non-trivial solution for the homogeneous equation, this set
    of vectors are not linearly independent.

\end{document}
