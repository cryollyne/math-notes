%%%%%%%%%%%%%%%%%%%%%%%%%%%%% Define Article %%%%%%%%%%%%%%%%%%%%%%%%%%%%%%%%%%
\documentclass{article}
%%%%%%%%%%%%%%%%%%%%%%%%%%%%%%%%%%%%%%%%%%%%%%%%%%%%%%%%%%%%%%%%%%%%%%%%%%%%%%%

%%%%%%%%%%%%%%%%%%%%%%%%%%%%% Using Packages %%%%%%%%%%%%%%%%%%%%%%%%%%%%%%%%%%
\usepackage{geometry}
\usepackage{graphicx}
\usepackage{amssymb}
\usepackage{amsmath}
\usepackage{amsthm}
\usepackage{empheq}
\usepackage{mdframed}
\usepackage{booktabs}
\usepackage{lipsum}
\usepackage{graphicx}
\usepackage{color}
\usepackage{psfrag}
\usepackage{pgfplots}
\usepackage{bm}
%%%%%%%%%%%%%%%%%%%%%%%%%%%%%%%%%%%%%%%%%%%%%%%%%%%%%%%%%%%%%%%%%%%%%%%%%%%%%%%

% Other Settings

%%%%%%%%%%%%%%%%%%%%%%%%%% Page Setting %%%%%%%%%%%%%%%%%%%%%%%%%%%%%%%%%%%%%%%
\geometry{a4paper}

%%%%%%%%%%%%%%%%%%%%%%%%%% Define some useful colors %%%%%%%%%%%%%%%%%%%%%%%%%%
\definecolor{ocre}{RGB}{243,102,25}
\definecolor{mygray}{RGB}{243,243,244}
\definecolor{deepGreen}{RGB}{26,111,0}
\definecolor{shallowGreen}{RGB}{235,255,255}
\definecolor{deepBlue}{RGB}{61,124,222}
\definecolor{shallowBlue}{RGB}{235,249,255}
%%%%%%%%%%%%%%%%%%%%%%%%%%%%%%%%%%%%%%%%%%%%%%%%%%%%%%%%%%%%%%%%%%%%%%%%%%%%%%%

%%%%%%%%%%%%%%%%%%%%%%%%%% Define an orangebox command %%%%%%%%%%%%%%%%%%%%%%%%
\newcommand\orangebox[1]{\fcolorbox{ocre}{mygray}{\hspace{1em}#1\hspace{1em}}}
%%%%%%%%%%%%%%%%%%%%%%%%%%%%%%%%%%%%%%%%%%%%%%%%%%%%%%%%%%%%%%%%%%%%%%%%%%%%%%%

%%%%%%%%%%%%%%%%%%%%%%%%%%%% English Environments %%%%%%%%%%%%%%%%%%%%%%%%%%%%%
\newtheoremstyle{mytheoremstyle}{3pt}{3pt}{\normalfont}{0cm}{\rmfamily\bfseries}{}{1em}{{\color{black}\thmname{#1}~\thmnumber{#2}}\thmnote{\,--\,#3}}
\newtheoremstyle{myproblemstyle}{3pt}{3pt}{\normalfont}{0cm}{\rmfamily\bfseries}{}{1em}{{\color{black}\thmname{#1}~\thmnumber{#2}}\thmnote{\,--\,#3}}
\theoremstyle{mytheoremstyle}
\newmdtheoremenv[linewidth=1pt,backgroundcolor=shallowGreen,linecolor=deepGreen,leftmargin=0pt,innerleftmargin=20pt,innerrightmargin=20pt,]{theorem}{Theorem}[section]
\theoremstyle{mytheoremstyle}
\newmdtheoremenv[linewidth=1pt,backgroundcolor=shallowBlue,linecolor=deepBlue,leftmargin=0pt,innerleftmargin=20pt,innerrightmargin=20pt,]{definition}{Definition}[section]
\theoremstyle{myproblemstyle}
\newmdtheoremenv[linecolor=black,leftmargin=0pt,innerleftmargin=10pt,innerrightmargin=10pt,]{problem}{Problem}[section]
%%%%%%%%%%%%%%%%%%%%%%%%%%%%%%%%%%%%%%%%%%%%%%%%%%%%%%%%%%%%%%%%%%%%%%%%%%%%%%%

%%%%%%%%%%%%%%%%%%%%%%%%%%%%%%% Plotting Settings %%%%%%%%%%%%%%%%%%%%%%%%%%%%%
\usepgfplotslibrary{colorbrewer}
\pgfplotsset{width=8cm,compat=1.9}
%%%%%%%%%%%%%%%%%%%%%%%%%%%%%%%%%%%%%%%%%%%%%%%%%%%%%%%%%%%%%%%%%%%%%%%%%%%%%%%

%%%%%%%%%%%%%%%%%%%%%%%%%%%%%%% Title & Author %%%%%%%%%%%%%%%%%%%%%%%%%%%%%%%%
\title{Matrix of a Linear Transformation}
\author{Patrick Chen}
\date{Sept 23, 2024}
%%%%%%%%%%%%%%%%%%%%%%%%%%%%%%%%%%%%%%%%%%%%%%%%%%%%%%%%%%%%%%%%%%%%%%%%%%%%%%%

\begin{document}
    \maketitle

    \subsection*{Linear Transformation}
    We say a transformation $T: \mathbb{R}^n \mapsto \mathbb{R}^m$ is linear
    if it follows the rules of linearity. All matrices are linear
    transformations.
    \begin{itemize}
        \item for all $u,v\in \mathbb{R}^n, T(u+v) = T(u)+T(v)$
        \item for all scalars $\lambda$ and $u\in \mathbb{R}^n, T(\lambda u) = \lambda T(u)$
    \end{itemize}

    Following these rules, we can conclude $T(0) = 0$ and for all scalars,
    $c,d$, and vectors $u,v$, $T(cu+dv) = cT(u) + dT(v)$

    \begin{align*}
        T(c_1u_1+\dots+c_ku_k) = c_1T(u_1)+\dots+c_kT(u_k)
    \end{align*}
    T's value on the span of $u_1\dots u_n$ is determined by its values at
    $u_1\dots u_n$

    \begin{align*}
        e_1 = \begin{bmatrix}
            1 \\ 0 \\ 0 \\ \vdots
        \end{bmatrix} &&
        e_1 = \begin{bmatrix}
            1 \\ 0 \\ 0 \\ \vdots
        \end{bmatrix} &&
        e_3 = \begin{bmatrix}
            0 \\ 0 \\ 1 \\ \vdots
        \end{bmatrix} && \dots \\
    \end{align*}
    \begin{align*}
        \begin{bmatrix}
            x_1 \\ \vdots \\ x_n
        \end{bmatrix} &=
        x_1e_1 + x_2e_2 + \dots + x_ne_n \\
        T\bigg(\begin{bmatrix}
            x_1 \\ \vdots \\ x_n
        \end{bmatrix}\bigg) &= T(x_1e_1+x_2e_2+\dots+x_ne_n) \\
        &= x_1T(e_1) + x_2T(e_2) + \dots + x_nT(e_n)
    \end{align*}
    A linear transformation is entirely defined by how it affects the basis
    vectors. All linear transformations from $T: \mathbb{R}^n \mapsto
    \mathbb{R}^m$ has a unique matrix that describes the transformation called
    the standard matrix. The standard matrix is composed of the transformations
    applied to the basis vectors.
    \begin{align*}
        A = \begin{bmatrix}
            T(e_1) & T(e_2) & T(e_3) & \dots
        \end{bmatrix}
    \end{align*}

    \subsection*{Common Transformations}
    Reflection on the $y=x$ line:
    \begin{align*}
        \begin{bmatrix}
            0 & 1 \\
            1 & 0
        \end{bmatrix}
    \end{align*}
    Counter clockwise rotation by the angle $\theta$:
    \begin{align*}
        \begin{bmatrix}
            \cos(\theta) & -\sin(\theta) \\
            \sin(\theta) & \cos(\theta)
        \end{bmatrix}
    \end{align*}

    \subsection*{Properties}
    \begin{itemize}
        \item A transformation is \textbf{one-to-one} if two different values of
            input always results in different outputs.

        \item A transform is \textbf{onto} if the range of a transformation is
            the same as the codomain
    \end{itemize}
    A linear transformation is one-to-one if the only answer to $T_A(x)=0$
    is the trivial solution $x=0$. This is equivalent to checking if the
    columns of the matrix is linearly independent.

\end{document}
