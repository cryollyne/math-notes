%%%%%%%%%%%%%%%%%%%%%%%%%%%%% Define Article %%%%%%%%%%%%%%%%%%%%%%%%%%%%%%%%%%
\documentclass{article}
%%%%%%%%%%%%%%%%%%%%%%%%%%%%%%%%%%%%%%%%%%%%%%%%%%%%%%%%%%%%%%%%%%%%%%%%%%%%%%%

%%%%%%%%%%%%%%%%%%%%%%%%%%%%% Using Packages %%%%%%%%%%%%%%%%%%%%%%%%%%%%%%%%%%
\usepackage{geometry}
\usepackage{graphicx}
\usepackage{amssymb}
\usepackage{amsmath}
\usepackage{amsthm}
\usepackage{empheq}
\usepackage{mdframed}
\usepackage{booktabs}
\usepackage{lipsum}
\usepackage{graphicx}
\usepackage{color}
\usepackage{psfrag}
\usepackage{pgfplots}
\usepackage{bm}
%%%%%%%%%%%%%%%%%%%%%%%%%%%%%%%%%%%%%%%%%%%%%%%%%%%%%%%%%%%%%%%%%%%%%%%%%%%%%%%

% Other Settings

%%%%%%%%%%%%%%%%%%%%%%%%%% Page Setting %%%%%%%%%%%%%%%%%%%%%%%%%%%%%%%%%%%%%%%
\geometry{a4paper}

%%%%%%%%%%%%%%%%%%%%%%%%%% Define some useful colors %%%%%%%%%%%%%%%%%%%%%%%%%%
\definecolor{ocre}{RGB}{243,102,25}
\definecolor{mygray}{RGB}{243,243,244}
\definecolor{deepGreen}{RGB}{26,111,0}
\definecolor{shallowGreen}{RGB}{235,255,255}
\definecolor{deepBlue}{RGB}{61,124,222}
\definecolor{shallowBlue}{RGB}{235,249,255}
%%%%%%%%%%%%%%%%%%%%%%%%%%%%%%%%%%%%%%%%%%%%%%%%%%%%%%%%%%%%%%%%%%%%%%%%%%%%%%%

%%%%%%%%%%%%%%%%%%%%%%%%%% Define an orangebox command %%%%%%%%%%%%%%%%%%%%%%%%
\newcommand\orangebox[1]{\fcolorbox{ocre}{mygray}{\hspace{1em}#1\hspace{1em}}}
%%%%%%%%%%%%%%%%%%%%%%%%%%%%%%%%%%%%%%%%%%%%%%%%%%%%%%%%%%%%%%%%%%%%%%%%%%%%%%%

%%%%%%%%%%%%%%%%%%%%%%%%%%%% English Environments %%%%%%%%%%%%%%%%%%%%%%%%%%%%%
\newtheoremstyle{mytheoremstyle}{3pt}{3pt}{\normalfont}{0cm}{\rmfamily\bfseries}{}{1em}{{\color{black}\thmname{#1}~\thmnumber{#2}}\thmnote{\,--\,#3}}
\newtheoremstyle{myproblemstyle}{3pt}{3pt}{\normalfont}{0cm}{\rmfamily\bfseries}{}{1em}{{\color{black}\thmname{#1}~\thmnumber{#2}}\thmnote{\,--\,#3}}
\theoremstyle{mytheoremstyle}
\newmdtheoremenv[linewidth=1pt,backgroundcolor=shallowGreen,linecolor=deepGreen,leftmargin=0pt,innerleftmargin=20pt,innerrightmargin=20pt,]{theorem}{Theorem}[section]
\theoremstyle{mytheoremstyle}
\newmdtheoremenv[linewidth=1pt,backgroundcolor=shallowBlue,linecolor=deepBlue,leftmargin=0pt,innerleftmargin=20pt,innerrightmargin=20pt,]{definition}{Definition}[section]
\theoremstyle{myproblemstyle}
\newmdtheoremenv[linecolor=black,leftmargin=0pt,innerleftmargin=10pt,innerrightmargin=10pt,]{problem}{Problem}[section]
%%%%%%%%%%%%%%%%%%%%%%%%%%%%%%%%%%%%%%%%%%%%%%%%%%%%%%%%%%%%%%%%%%%%%%%%%%%%%%%

%%%%%%%%%%%%%%%%%%%%%%%%%%%%%%% Plotting Settings %%%%%%%%%%%%%%%%%%%%%%%%%%%%%
\usepgfplotslibrary{colorbrewer}
\pgfplotsset{width=8cm,compat=1.9}
%%%%%%%%%%%%%%%%%%%%%%%%%%%%%%%%%%%%%%%%%%%%%%%%%%%%%%%%%%%%%%%%%%%%%%%%%%%%%%%

%%%%%%%%%%%%%%%%%%%%%%%%%%%%%%% Title & Author %%%%%%%%%%%%%%%%%%%%%%%%%%%%%%%%
\title{Inverse of a Matrix}
\author{Patrick Chen}
\date{Oct 1, 2024}
%%%%%%%%%%%%%%%%%%%%%%%%%%%%%%%%%%%%%%%%%%%%%%%%%%%%%%%%%%%%%%%%%%%%%%%%%%%%%%%

\begin{document}
    \maketitle
    \section*{Powers of a matrix}
    When a square matrix is raised to a power $A^n$, this is equivalent to
    multiplying by the matrix $n$ times. When a matrix is raised to the zeroth
    power, it is equal to the identity matrix.
    \begin{align*}
        A^2 &=AA \\
        A^n &= AA^{n-1} \\
        A^0 &= I
    \end{align*}

    \section*{Transpose}
    The transpose of a matrix is the matrix flipped along the main diagonal.
    \begin{align*}
        (A+B)^T         &= A^T+B^T \\
        (AB)^T          &= B^TA^T \\
        (\lambda A)^T   &= \lambda A^T \\
        T_{A}: \mathbb{R}^n \mapsto \mathbb{R}^m &\Leftrightarrow
        T_{A^T}: \mathbb{R}^m \mapsto \mathbb{R}^n \\
        T_{A} \circ T_{A^T}: \mathbb{R}^n \mapsto \mathbb{R}^n &\Leftrightarrow
        T_{A^T} \circ T_{A}: \mathbb{R}^m \mapsto \mathbb{R}^m \\
    \end{align*}
    A matrix is called symmetric if $A=A^T$

    \begin{align*}
        A = \begin{bmatrix}
            2 & 0 \\
            1 & 3
        \end{bmatrix} &&
        A^T = \begin{bmatrix}
            2 & 1 \\
            0 & 3
        \end{bmatrix} \\
        B = \begin{bmatrix}
            1 & 0 & 4 \\
            -1 & 3 & 1
        \end{bmatrix} &&
        B^T = \begin{bmatrix}
            1 & -1 \\
            0 & 3 \\
            4 & 1
        \end{bmatrix}
    \end{align*}

    \section*{Inverse of a matrix}
    We say a $n\times n$ matrix $A$ is invertible if there exists a matrix $B$
    such that $AB=BA=I$. The matrix $B$ is the inverse of $A$ and can be written
    as $A^{-1}$. If $A$ is invertible, then $A$ is one-to-one and onto. The
    inverse of a matrix is always unique if it exists.

    \begin{align*}
        BAC = B(AC) &= BI = B \\
        BAC = (BA)C &= IC = C \\
        BAC = B &= C
    \end{align*}

    The inverse of the inverse of matrix $A$ is $A$.
    \begin{align*}
        (A^{-1})^{-1} = A \\
        (A^{-1})^{-1} = B \\
        A^{-1}(A^{-1})^{-1} = A^{-1}B \\
        I = A^{-1}B \\
        AI = AA^{-1}B \\
        A = B \\
    \end{align*}

    When two matrices are multiplied, then inverted, it is equivalent to
    reversing the order of matrices and inverting all of them.
    \begin{align*}
        (ABC)^{-1} &= C^{-1}B^{-1}A^{-1} \\
        (ABC)(ABC)^{-1} &= (ABC)(C^{-1}B^{-1}A^{-1}) \\
        I &= AB(CC^{-1})B^{-1}A^{-1} \\
        I &= ABIB^{-1}A^{-1} \\
        I &= ABB^{-1}A^{-1} \\
        I &= AIA^{-1} \\
        I &= AA^{-1} \\
        I &= I
    \end{align*}

    Transpose and invert commute with each other.
    \begin{align*}
        (A^T)^{-1} = (A^{-1})^T = A^{-T}
    \end{align*}

    For a two by two matrix, If $ad-bc = 0$, then the matrix is not invertible.
    \begin{align*}
        \begin{bmatrix}
            a & b \\
            c & d
        \end{bmatrix}
        \begin{bmatrix}
            d & -b \\
            -c & a
        \end{bmatrix} &=
        \begin{bmatrix}
            ad-bc & 0 \\
            0 & ad-bc
        \end{bmatrix} \\
        \begin{bmatrix}
            a & b \\
            c & d
        \end{bmatrix}^{-1} &=
        \frac{1}{ad-bc} \begin{bmatrix}
            d & -b \\
            -c & a
        \end{bmatrix}
    \end{align*}
    \begin{align*}
        ad-bc&=0 \\
        ad &= cd \\
        \frac{a}{c} &= \frac{b}{d}
    \end{align*}
    Note that this means that ac and bd are linearly dependent.

    Suppose $A$ is an invertible matrix
    \begin{align*}
        Ax=0 \\
        A^{-1}Ax = A^{-1}0 \\
        x = 0
    \end{align*}
    Since there is only one solution to the homogeneous equation, A is
    one-to-one.
    \begin{align*}
        Ax &= b \\
        A^{-1}Ax &= A^{-1}b \\
        x &= A^{-1}b
    \end{align*}
    Since all vectors in $\mathbb{R}^n$ has a vector in $\mathbb{R}^n$ that maps
    to it, $T_{A}: \mathbb{R}^n \mapsto \mathbb{R}^n$ is onto.

    \section*{Elementary Matrices}
    A elementary matrix is one row operation applied to the identity matrix.
    \begin{itemize}
        \item For interchanging rows, the matrix $E$ will be its own inverse
            because swapping a row twice is the same as not swapping at all.
            Multiplying by this matrix will interchange rows on the matrix being
            multiplied.
            \begin{align*}
                \begin{bmatrix}
                    0 & 1 & 0 \\
                    1 & 0 & 0 \\
                    0 & 0 & 1 \\
                \end{bmatrix}^{-1} = 
                \begin{bmatrix}
                    0 & 1 & 0 \\
                    1 & 0 & 0 \\
                    0 & 0 & 1 \\
                \end{bmatrix}
            \end{align*}

        \item For multiplying by a non zero scalar, the inverse of matrix $E$
            will be the identity matrix with the row multiplied by the
            reciprocal of the multiplied scalar. Multiplying by this matrix will
            result in a row being multiplied by the non zero scalar.
            \begin{align*}
                \begin{bmatrix}
                    1 & 0 & 0 \\
                    0 & 2 & 0 \\
                    0 & 0 & 1 \\
                \end{bmatrix}^{-1} = 
                \begin{bmatrix}
                    1 & 0 & 0 \\
                    0 & \frac{1}{2} & 0 \\
                    0 & 0 & 1 \\
                \end{bmatrix} \\
            \end{align*}

        \item For adding a multiple of another row, the inverse of $E$ will be
            the that row subtracted by the multiply of the row. Multiplying by
            this matrix will result in a multiply a row being added to the row.
            \begin{align*}
                \begin{bmatrix}
                    1 & 0 & 4 \\
                    0 & 1 & 0 \\
                    0 & 0 & 1 \\
                \end{bmatrix}^{-1} = 
                \begin{bmatrix}
                    1 & 0 & -4 \\
                    0 & 1 & 0 \\
                    0 & 0 & 1 \\
                \end{bmatrix} \\
            \end{align*}
    \end{itemize}

    Left multiplying by elementary matrices is identical to preforming row
    operations. If by left multiplying elementary matrices, some matrix $A$ can
    be transformed into the identity matrix, then the series of elementary
    matrices is the inverse of $A$.

    \begin{align*}
        E_n \dots E_2E_1 A &= I \\
        E_n \dots E_2E_1 I &= A^{-1} \\
    \end{align*}

    If the matrix $[A\ I]$ is row reduced, and $A$ is invertible, then the
    reduced row echelon form will be $[I\ A^{-1}]$. If $A$ is not invertible,
    then the row reduced form will not contain the identity matrix.

\end{document}
