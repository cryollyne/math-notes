%%%%%%%%%%%%%%%%%%%%%%%%%%%%% Define Article %%%%%%%%%%%%%%%%%%%%%%%%%%%%%%%%%%
\documentclass{article}
%%%%%%%%%%%%%%%%%%%%%%%%%%%%%%%%%%%%%%%%%%%%%%%%%%%%%%%%%%%%%%%%%%%%%%%%%%%%%%%

%%%%%%%%%%%%%%%%%%%%%%%%%%%%% Using Packages %%%%%%%%%%%%%%%%%%%%%%%%%%%%%%%%%%
\usepackage{geometry}
\usepackage{graphicx}
\usepackage{amssymb}
\usepackage{amsmath}
\usepackage{amsthm}
\usepackage{empheq}
\usepackage{mdframed}
\usepackage{booktabs}
\usepackage{lipsum}
\usepackage{graphicx}
\usepackage{color}
\usepackage{psfrag}
\usepackage{pgfplots}
\usepackage{bm}
%%%%%%%%%%%%%%%%%%%%%%%%%%%%%%%%%%%%%%%%%%%%%%%%%%%%%%%%%%%%%%%%%%%%%%%%%%%%%%%

% Other Settings

%%%%%%%%%%%%%%%%%%%%%%%%%% Page Setting %%%%%%%%%%%%%%%%%%%%%%%%%%%%%%%%%%%%%%%
\geometry{a4paper}

%%%%%%%%%%%%%%%%%%%%%%%%%% Define some useful colors %%%%%%%%%%%%%%%%%%%%%%%%%%
\definecolor{ocre}{RGB}{243,102,25}
\definecolor{mygray}{RGB}{243,243,244}
\definecolor{deepGreen}{RGB}{26,111,0}
\definecolor{shallowGreen}{RGB}{235,255,255}
\definecolor{deepBlue}{RGB}{61,124,222}
\definecolor{shallowBlue}{RGB}{235,249,255}
%%%%%%%%%%%%%%%%%%%%%%%%%%%%%%%%%%%%%%%%%%%%%%%%%%%%%%%%%%%%%%%%%%%%%%%%%%%%%%%

%%%%%%%%%%%%%%%%%%%%%%%%%% Define an orangebox command %%%%%%%%%%%%%%%%%%%%%%%%
\newcommand\orangebox[1]{\fcolorbox{ocre}{mygray}{\hspace{1em}#1\hspace{1em}}}
%%%%%%%%%%%%%%%%%%%%%%%%%%%%%%%%%%%%%%%%%%%%%%%%%%%%%%%%%%%%%%%%%%%%%%%%%%%%%%%

%%%%%%%%%%%%%%%%%%%%%%%%%%%% English Environments %%%%%%%%%%%%%%%%%%%%%%%%%%%%%
\newtheoremstyle{mytheoremstyle}{3pt}{3pt}{\normalfont}{0cm}{\rmfamily\bfseries}{}{1em}{{\color{black}\thmname{#1}~\thmnumber{#2}}\thmnote{\,--\,#3}}
\newtheoremstyle{myproblemstyle}{3pt}{3pt}{\normalfont}{0cm}{\rmfamily\bfseries}{}{1em}{{\color{black}\thmname{#1}~\thmnumber{#2}}\thmnote{\,--\,#3}}
\theoremstyle{mytheoremstyle}
\newmdtheoremenv[linewidth=1pt,backgroundcolor=shallowGreen,linecolor=deepGreen,leftmargin=0pt,innerleftmargin=20pt,innerrightmargin=20pt,]{theorem}{Theorem}[section]
\theoremstyle{mytheoremstyle}
\newmdtheoremenv[linewidth=1pt,backgroundcolor=shallowBlue,linecolor=deepBlue,leftmargin=0pt,innerleftmargin=20pt,innerrightmargin=20pt,]{definition}{Definition}[section]
\theoremstyle{myproblemstyle}
\newmdtheoremenv[linecolor=black,leftmargin=0pt,innerleftmargin=10pt,innerrightmargin=10pt,]{problem}{Problem}[section]
%%%%%%%%%%%%%%%%%%%%%%%%%%%%%%%%%%%%%%%%%%%%%%%%%%%%%%%%%%%%%%%%%%%%%%%%%%%%%%%

%%%%%%%%%%%%%%%%%%%%%%%%%%%%%%% Plotting Settings %%%%%%%%%%%%%%%%%%%%%%%%%%%%%
\usepgfplotslibrary{colorbrewer}
\pgfplotsset{width=8cm,compat=1.9}
%%%%%%%%%%%%%%%%%%%%%%%%%%%%%%%%%%%%%%%%%%%%%%%%%%%%%%%%%%%%%%%%%%%%%%%%%%%%%%%

%%%%%%%%%%%%%%%%%%%%%%%%%%%%%%% Title & Author %%%%%%%%%%%%%%%%%%%%%%%%%%%%%%%%
\title{Matrix Algebra}
\author{Patrick Chen}
\date{Sept 26, 2024}
%%%%%%%%%%%%%%%%%%%%%%%%%%%%%%%%%%%%%%%%%%%%%%%%%%%%%%%%%%%%%%%%%%%%%%%%%%%%%%%

\begin{document}
    \maketitle
    \section*{Addition and multiplication}
    When two matrices of the same size are added or subtracted, it is done
    element-wise.
    \begin{align*}
        \begin{bmatrix}
            A_{11} & A_{12} & \dots \\
            A_{21} & A_{22} & \\
            \vdots &        & \ddots \\
        \end{bmatrix} +
        \begin{bmatrix}
            B_{11} & B_{12} & \dots \\
            B_{21} & B_{22} & \\
            \vdots &        & \ddots \\
        \end{bmatrix} =
        \begin{bmatrix}
            A_{11}+B_{11} & A_{12}+B_{12} & \dots \\
            A_{21}+B_{21} & A_{22}+B_{22} & \\
            \vdots        &               & \ddots \\
        \end{bmatrix}
    \end{align*}

    When a matrix is multiplied by a scalar, every entry in the matrix is
    multiplied by the scalar
    \begin{align*}
        \lambda \begin{bmatrix}
            A_{11} & A_{12} & A_{13} & \\
            A_{21} & A_{22} & A_{23} & \dots \\
            A_{31} & A_{32} & A_{33} & \\
                   & \vdots &        & \ddots \\
        \end{bmatrix} =
        \begin{bmatrix}
            \lambda A_{11} & \lambda A_{12} & \lambda A_{13} & \\
            \lambda A_{21} & \lambda A_{22} & \lambda A_{23} & \dots \\
            \lambda A_{31} & \lambda A_{32} & \lambda A_{33} & \\
                   & \vdots &        & \ddots \\
        \end{bmatrix}
    \end{align*}

    \begin{align*}
        A+B             &= B+A & \text{matrix addition commutes}\\
        \lambda (A+B)   &= \lambda A + \lambda B & \text{scalars are distributive}\\
        (\lambda\mu)A   &= \lambda(\mu A)
    \end{align*}

    \section*{Matrix Multiplication}
    When $A$ is a $m\times k$ matrix and $B$ is a $k\times n$ matrix, we define
    $AB$ as:
    \begin{align*}
        \begin{bmatrix}
            Ab_1 & Ab_2 & \dots & AB_n
        \end{bmatrix}
    \end{align*}
    Note that multiplication of matrices do not commute. That means that $AB\ne
    BA$ When $A$ and $B$ are multiplied, this is equivalent to composing their
    transformations. $AB \equiv T_A \circ T_B$

    exmaple:
    \begin{align*}
        \begin{bmatrix}
            2 & 1 & 3 \\
            0 & -1 & -2
        \end{bmatrix} \begin{bmatrix}
            1 & 2 \\
            0 & -1 \\
            1 & 1
        \end{bmatrix} = \begin{bmatrix}
            5 & 6 \\
            -2 & -1
        \end{bmatrix}
    \end{align*}

    \begin{align*}
        \begin{bmatrix}
            1 & 1 \\
            0 & 1
        \end{bmatrix}
        \begin{bmatrix}
            1 & 0 \\
            1 & 1
        \end{bmatrix} = 
        \begin{bmatrix}
            2 & 1 \\
            1 & 1
        \end{bmatrix} \\
        \begin{bmatrix}
            1 & 0 \\
            1 & 1
        \end{bmatrix}
        \begin{bmatrix}
            1 & 1 \\
            0 & 1
        \end{bmatrix} = 
        \begin{bmatrix}
            1 & 1 \\
            1 & 2
        \end{bmatrix}
    \end{align*}

    \subsection*{Properties of matrix algebra}
    \begin{itemize}
        \item Associativity $A(BC) = (AB)C$
        \item Left distributivity $A(B+C) = AB+AC$
        \item Right distributivity $(A+B)C = AC+BC$
        \item Scalar commutativity $\lambda(AB) = (\lambda A)B = A(\lambda B)$
        \item Identity $IA = AI = A$
    \end{itemize}
    Matrices are associative because multiplying by a matrix corresponds to
    linear transformations. Since function composition is associative, so is
    matrix multiplication. Distributivity is split between left and right
    distributivity because multiplication does not commute. Thus, multiplying
    from the left will distribute the matrix with the multiplication on the
    left, and vise versa for right.

    \subsection*{Identity and Zero Matrix}
    A matrix composed with the basis vectors is called the identity matrix. This
    matrix is special because left multiplication and right multiplication by
    this matrix is equivalent to not doing anything.
    \begin{align*}
        I = \begin{bmatrix}
            1 & 0 & 0 & \\
            0 & 1 & 0 & \dots \\
            0 & 0 & 1 & \\
             & \vdots &  & \ddots
        \end{bmatrix}
    \end{align*}

    When a matrix is composed entirely of zeros, it is called the zero matrix
    because it acts like a zero in addition.
    \begin{align*}
        0 = \begin{bmatrix}
            0 & 0 & \dots \\
            0 & 0 & \\
            \vdots & & \ddots
        \end{bmatrix}
    \end{align*}
\end{document}
