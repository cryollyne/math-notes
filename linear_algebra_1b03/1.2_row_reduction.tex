%%%%%%%%%%%%%%%%%%%%%%%%%%%%% Define Article %%%%%%%%%%%%%%%%%%%%%%%%%%%%%%%%%%
\documentclass{article}
%%%%%%%%%%%%%%%%%%%%%%%%%%%%%%%%%%%%%%%%%%%%%%%%%%%%%%%%%%%%%%%%%%%%%%%%%%%%%%%

%%%%%%%%%%%%%%%%%%%%%%%%%%%%% Using Packages %%%%%%%%%%%%%%%%%%%%%%%%%%%%%%%%%%
\usepackage{geometry}
\usepackage{graphicx}
\usepackage{amssymb}
\usepackage{amsmath}
\usepackage{amsthm}
\usepackage{empheq}
\usepackage{mdframed}
\usepackage{booktabs}
\usepackage{lipsum}
\usepackage{graphicx}
\usepackage{color}
\usepackage{psfrag}
\usepackage{pgfplots}
\usepackage{bm}
%%%%%%%%%%%%%%%%%%%%%%%%%%%%%%%%%%%%%%%%%%%%%%%%%%%%%%%%%%%%%%%%%%%%%%%%%%%%%%%

% Other Settings

%%%%%%%%%%%%%%%%%%%%%%%%%% Page Setting %%%%%%%%%%%%%%%%%%%%%%%%%%%%%%%%%%%%%%%
\geometry{a4paper}

%%%%%%%%%%%%%%%%%%%%%%%%%% Define some useful colors %%%%%%%%%%%%%%%%%%%%%%%%%%
\definecolor{ocre}{RGB}{243,102,25}
\definecolor{mygray}{RGB}{243,243,244}
\definecolor{deepGreen}{RGB}{26,111,0}
\definecolor{shallowGreen}{RGB}{235,255,255}
\definecolor{deepBlue}{RGB}{61,124,222}
\definecolor{shallowBlue}{RGB}{235,249,255}
%%%%%%%%%%%%%%%%%%%%%%%%%%%%%%%%%%%%%%%%%%%%%%%%%%%%%%%%%%%%%%%%%%%%%%%%%%%%%%%

%%%%%%%%%%%%%%%%%%%%%%%%%% Define an orangebox command %%%%%%%%%%%%%%%%%%%%%%%%
\newcommand\orangebox[1]{\fcolorbox{ocre}{mygray}{\hspace{1em}#1\hspace{1em}}}
%%%%%%%%%%%%%%%%%%%%%%%%%%%%%%%%%%%%%%%%%%%%%%%%%%%%%%%%%%%%%%%%%%%%%%%%%%%%%%%

%%%%%%%%%%%%%%%%%%%%%%%%%%%% English Environments %%%%%%%%%%%%%%%%%%%%%%%%%%%%%
\newtheoremstyle{mytheoremstyle}{3pt}{3pt}{\normalfont}{0cm}{\rmfamily\bfseries}{}{1em}{{\color{black}\thmname{#1}~\thmnumber{#2}}\thmnote{\,--\,#3}}
\newtheoremstyle{myproblemstyle}{3pt}{3pt}{\normalfont}{0cm}{\rmfamily\bfseries}{}{1em}{{\color{black}\thmname{#1}~\thmnumber{#2}}\thmnote{\,--\,#3}}
\theoremstyle{mytheoremstyle}
\newmdtheoremenv[linewidth=1pt,backgroundcolor=shallowGreen,linecolor=deepGreen,leftmargin=0pt,innerleftmargin=20pt,innerrightmargin=20pt,]{theorem}{Theorem}[section]
\theoremstyle{mytheoremstyle}
\newmdtheoremenv[linewidth=1pt,backgroundcolor=shallowBlue,linecolor=deepBlue,leftmargin=0pt,innerleftmargin=20pt,innerrightmargin=20pt,]{definition}{Definition}[section]
\theoremstyle{myproblemstyle}
\newmdtheoremenv[linecolor=black,leftmargin=0pt,innerleftmargin=10pt,innerrightmargin=10pt,]{problem}{Problem}[section]
%%%%%%%%%%%%%%%%%%%%%%%%%%%%%%%%%%%%%%%%%%%%%%%%%%%%%%%%%%%%%%%%%%%%%%%%%%%%%%%

%%%%%%%%%%%%%%%%%%%%%%%%%%%%%%% Plotting Settings %%%%%%%%%%%%%%%%%%%%%%%%%%%%%
\usepgfplotslibrary{colorbrewer}
\pgfplotsset{width=8cm,compat=1.9}
%%%%%%%%%%%%%%%%%%%%%%%%%%%%%%%%%%%%%%%%%%%%%%%%%%%%%%%%%%%%%%%%%%%%%%%%%%%%%%%

%%%%%%%%%%%%%%%%%%%%%%%%%%%%%%% Title & Author %%%%%%%%%%%%%%%%%%%%%%%%%%%%%%%%
\title{Row Reduction}
\author{Patrick Chen}
\date{Sept 5, 2024}
%%%%%%%%%%%%%%%%%%%%%%%%%%%%%%%%%%%%%%%%%%%%%%%%%%%%%%%%%%%%%%%%%%%%%%%%%%%%%%%

\begin{document}
    \maketitle

    Finding solutions to systems of linear equations:
    \begin{align*}
        \begin{matrix}
               & & 2y &+& 3z &=& -4 \\
            2x &+& 6y &+& 2z &=& 4 \\
            3x & &    &+& z  &=& 3
        \end{matrix}
    \end{align*}

    An augmented matrix representing the system of linear equations:
    \begin{align*}
        \begin{bmatrix}
            0 & 2 & 2 & -4 \\
            2 & 6 & 2 & 4 \\
            3 & 0 & 1 & 3 \\
        \end{bmatrix}
    \end{align*}

    The coefficient matrix representing the system of linear equations:
    \begin{align*}
        \begin{bmatrix}
            0 & 2 & 2  \\
            2 & 6 & 2  \\
            3 & 0 & 1  \\
        \end{bmatrix}
    \end{align*}

    Rules for manipulating augmented matrices
    \begin{itemize}
        \item interchange rows
        \item multiply an entire row by a non-zero constant
        \item add a constant multiple of another row
    \end{itemize}

    The row reduction algorithm is as follows:
    \begin{enumerate}
        \item divide the first row by the first element, causing the first
            element to be one. This will make the first row have the form:
        \begin{align*}
            \begin{bmatrix}
                1 & a_{12} & a_{13} & \ldots & a_{1m} & b_1
            \end{bmatrix}
        \end{align*}

        \item subtract multiples of the first row from every row below such that
        the first element of those rows is zero. This should make the matrix
        take the following form:
        \begin{align*}
            \begin{bmatrix}
                1 & a_{12} & a_{13} &        & a_{1m} & b_1 \\
                0 & a_{22} & a_{23} & \ldots & a_{2m} & b_2 \\
                0 & a_{32} & a_{33} &        & a_{3m} & b_3 \\
                  & \vdots &        & \ddots &        &     \\
                0 & a_{n2} & a_{n3} &        & a_{nm} & b_n
            \end{bmatrix}
        \end{align*}

        \item repeat for the second element, then the third, etc. afterward the
        matrix should take the form:
        \begin{align*}
            \begin{bmatrix}
                1 & a_{12} & a_{13} &        & a_{1m} & b_1 \\
                0 & 1      & a_{23} & \ldots & a_{2m} & b_2 \\
                0 & 0      & 1      &        & a_{3m} & b_3 \\
                  & \vdots &        & \ddots &        &     \\
                0 & 0      & 0      &        & 1      & b_n
            \end{bmatrix}
        \end{align*}

        \item subtract multiples of the second row from each row above such the
            second column results in zero. Repeat for third row and third
            column, then fourth, etc. This should result in the matrix taking
            the form:
        \begin{align*}
            \begin{bmatrix}
                1 & 0      & 0      &        & 0      & b_1 \\
                0 & 1      & 0      & \ldots & 0      & b_2 \\
                0 & 0      & 1      &        & 0      & b_3 \\
                  & \vdots &        & \ddots &        &     \\
                0 & 0      & 0      &        & 1      & b_n
            \end{bmatrix}
        \end{align*}
    \end{enumerate}

    Row Reduction example
    \begin{align*}
        \begin{bmatrix}
            0 & 2 & 2 & -4 \\
            2 & 6 & 2 & 4 \\
            3 & 0 & 1 & 3 \\
        \end{bmatrix}
        &=\begin{bmatrix}
            2 & 6 & 2 & 4 \\
            0 & 2 & 2 & -4 \\
            3 & 0 & 1 & 3 \\
        \end{bmatrix} \text{swap row (1), (2)} \\
        &=\begin{bmatrix}
            1 & 3 & 1 & 2 \\
            0 & 2 & 2 & -4 \\
            3 & 0 & 1 & 3 \\
        \end{bmatrix} \text{divide row (1) by 2} \\
        &=\begin{bmatrix}
            1 & 3 & 1 & 2 \\
            0 & 2 & 2 & -4 \\
            0 &-9 &-2 &-3 \\
        \end{bmatrix}  \text{subtract 3 $\times$ row (1) from row (3)}\\
        &=\begin{bmatrix}
            1 & 3 & 1 & 2 \\
            0 & 1 & 1 & -2 \\
            0 &-9 &-2 &-3 \\
        \end{bmatrix} \text{divide row (2) by 2}\\
        &=\begin{bmatrix}
            1 & 3 & 1 & 2 \\
            0 & 1 & 1 & -2 \\
            0 & 0 & 7 & -21 \\
        \end{bmatrix} \text{add 9 $\times$ row (2) to row (3)}\\
        &=\begin{bmatrix}
            1 & 3 & 1 & 2 \\
            0 & 1 & 1 & -2 \\
            0 & 0 & 1 & -3 \\
        \end{bmatrix} \text{divide row (3) by 7}\\
        &=\begin{bmatrix}
            1 & 0 &-2 & 8 \\
            0 & 1 & 1 & -2 \\
            0 & 0 & 1 & -3 \\
        \end{bmatrix} \text{subtract 3 $\times$ row (2) from row (1)}\\
        &=\begin{bmatrix}
            1 & 0 & 0 & 2 \\
            0 & 1 & 0 & 1 \\
            0 & 0 & 1 & -3 \\
        \end{bmatrix} \text{add 2 $\times$ row (3) to row (1),
                            subtract row (3) from row (2)}\\
    \end{align*}
    \pagebreak

    A matrix in is row echelon form if:
    \begin{enumerate}
        \item All zero rows are grouped contiguously at the bottom of the matrix

        \item In any two consecutive row which are both non-zero, the non-zero
            entry of the upper row is to the left of the non-zero entry of the
            lower row
    \end{enumerate}
    If all leading entries in non-zero rows are 1, it is in \textbf{reduced} row
    echelon form.

    \section*{Example 2}
    \begin{equation*}
        \begin{matrix}
               & & y &+& 3z &=& 4 \\
            2x &-& y &+& z  &=& 2
        \end{matrix}
    \end{equation*}

    \begin{align*}
        \begin{bmatrix}
            0 & 1 & 3 & 4 \\
            2 &-1 & 1 & 2
        \end{bmatrix} \\
        \begin{bmatrix}
            2 &-1 & 1 & 2 \\
            0 & 1 & 3 & 4
        \end{bmatrix} \\
        \begin{bmatrix}
            1 &- \frac{1}{2} & \frac{1}{2} & 1 \\
            0 & 1 & 3 & 4
        \end{bmatrix} \\
        \begin{bmatrix}
            1 & 0 & 2 & 3 \\
            0 & 1 & 3 & 4
        \end{bmatrix} \\
        \begin{matrix}
            x && &+& 2z&=&3 \\
              &&y&+& 3z&=&4
        \end{matrix} \\
    \end{align*}
    \begin{align*}
        \therefore x =& 3-2t \\
                   y =& 4-3t \\
                   z =& t \\
                      &\text{where } t \in \mathbb{R}
    \end{align*}

    Since there is one free variable, there are infinite solutions residing on a
    line. If there are two free variables, there will be infinite solutions
    residing on a plane. For three free variables, a volume. etc. \\
    Sometimes a matrix may have no solutions. The following matrix has the row
    $\begin{bmatrix} 0 & 0 & 0 & 1 \end{bmatrix}$, which is equivalent to the
    non-sense expression $0x+0y+0z=1$. 
    \begin{equation*}
        \begin{bmatrix}
            1 & 0 & 1 & 2 \\
            0 & 0 & 1 & 4 \\
            0 & 0 & 0 & 1
        \end{bmatrix}
    \end{equation*}


\end{document}
