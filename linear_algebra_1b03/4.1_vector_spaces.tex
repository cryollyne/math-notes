%%%%%%%%%%%%%%%%%%%%%%%%%%%%% Define Article %%%%%%%%%%%%%%%%%%%%%%%%%%%%%%%%%%
\documentclass{article}
%%%%%%%%%%%%%%%%%%%%%%%%%%%%%%%%%%%%%%%%%%%%%%%%%%%%%%%%%%%%%%%%%%%%%%%%%%%%%%%

%%%%%%%%%%%%%%%%%%%%%%%%%%%%% Using Packages %%%%%%%%%%%%%%%%%%%%%%%%%%%%%%%%%%
\usepackage{geometry}
\usepackage{graphicx}
\usepackage{amssymb}
\usepackage{amsmath}
\usepackage{amsthm}
\usepackage{empheq}
\usepackage{mdframed}
\usepackage{booktabs}
\usepackage{lipsum}
\usepackage{graphicx}
\usepackage{color}
\usepackage{psfrag}
\usepackage{pgfplots}
\usepackage{bm}
%%%%%%%%%%%%%%%%%%%%%%%%%%%%%%%%%%%%%%%%%%%%%%%%%%%%%%%%%%%%%%%%%%%%%%%%%%%%%%%

% Other Settings

%%%%%%%%%%%%%%%%%%%%%%%%%% Page Setting %%%%%%%%%%%%%%%%%%%%%%%%%%%%%%%%%%%%%%%
\geometry{a4paper}

%%%%%%%%%%%%%%%%%%%%%%%%%% Define some useful colors %%%%%%%%%%%%%%%%%%%%%%%%%%
\definecolor{ocre}{RGB}{243,102,25}
\definecolor{mygray}{RGB}{243,243,244}
\definecolor{deepGreen}{RGB}{26,111,0}
\definecolor{shallowGreen}{RGB}{235,255,255}
\definecolor{deepBlue}{RGB}{61,124,222}
\definecolor{shallowBlue}{RGB}{235,249,255}
%%%%%%%%%%%%%%%%%%%%%%%%%%%%%%%%%%%%%%%%%%%%%%%%%%%%%%%%%%%%%%%%%%%%%%%%%%%%%%%

%%%%%%%%%%%%%%%%%%%%%%%%%% Define an orangebox command %%%%%%%%%%%%%%%%%%%%%%%%
\newcommand\orangebox[1]{\fcolorbox{ocre}{mygray}{\hspace{1em}#1\hspace{1em}}}
%%%%%%%%%%%%%%%%%%%%%%%%%%%%%%%%%%%%%%%%%%%%%%%%%%%%%%%%%%%%%%%%%%%%%%%%%%%%%%%

%%%%%%%%%%%%%%%%%%%%%%%%%%%% English Environments %%%%%%%%%%%%%%%%%%%%%%%%%%%%%
\newtheoremstyle{mytheoremstyle}{3pt}{3pt}{\normalfont}{0cm}{\rmfamily\bfseries}{}{1em}{{\color{black}\thmname{#1}~\thmnumber{#2}}\thmnote{\,--\,#3}}
\newtheoremstyle{myproblemstyle}{3pt}{3pt}{\normalfont}{0cm}{\rmfamily\bfseries}{}{1em}{{\color{black}\thmname{#1}~\thmnumber{#2}}\thmnote{\,--\,#3}}
\theoremstyle{mytheoremstyle}
\newmdtheoremenv[linewidth=1pt,backgroundcolor=shallowGreen,linecolor=deepGreen,leftmargin=0pt,innerleftmargin=20pt,innerrightmargin=20pt,]{theorem}{Theorem}[section]
\theoremstyle{mytheoremstyle}
\newmdtheoremenv[linewidth=1pt,backgroundcolor=shallowBlue,linecolor=deepBlue,leftmargin=0pt,innerleftmargin=20pt,innerrightmargin=20pt,]{definition}{Definition}[section]
\theoremstyle{myproblemstyle}
\newmdtheoremenv[linecolor=black,leftmargin=0pt,innerleftmargin=10pt,innerrightmargin=10pt,]{problem}{Problem}[section]
%%%%%%%%%%%%%%%%%%%%%%%%%%%%%%%%%%%%%%%%%%%%%%%%%%%%%%%%%%%%%%%%%%%%%%%%%%%%%%%

%%%%%%%%%%%%%%%%%%%%%%%%%%%%%%% Plotting Settings %%%%%%%%%%%%%%%%%%%%%%%%%%%%%
\usepgfplotslibrary{colorbrewer}
\pgfplotsset{width=8cm,compat=1.9}
%%%%%%%%%%%%%%%%%%%%%%%%%%%%%%%%%%%%%%%%%%%%%%%%%%%%%%%%%%%%%%%%%%%%%%%%%%%%%%%

%%%%%%%%%%%%%%%%%%%%%%%%%%%%%%% Title & Author %%%%%%%%%%%%%%%%%%%%%%%%%%%%%%%%
\title{Vector Spaces}
\author{Patrick Chen}
\date{Oct 28, 2024}
%%%%%%%%%%%%%%%%%%%%%%%%%%%%%%%%%%%%%%%%%%%%%%%%%%%%%%%%%%%%%%%%%%%%%%%%%%%%%%%

\newcommand\z{\pmb{\vec{0}}}
\newcommand\p{\pmb{+}}

\begin{document}
    \maketitle
    A vector space $V$ is a non-empty set together with a binary function $+$
    and unary functions $c$ for every $c\in \mathbb{R}$ satisfying the following
    axioms. Here, bold plus ($\p$) represents addition between the abstract
    objects in the set, regular plus ($+$) represents addition between the
    scalars and zero ($\z$) represents a abstract zero element in the set that
    behaves like a additive identity. The abstract addition operator is often
    just regular addition but it doesn't have to be. The same is true for the
    zero element.

    \begin{enumerate}
        \item The vector space is closed under addition
        \item Addition is associative
        \item Addition is commutative
        \item there exists $\z \in V$ such that $v\p\z=v$ for all $v\in V$
        \item for every $v\in V$, there exists $-v\in V$ such that $v\p(-v)=\z$
        \item for every $c\in \mathbb{R}$ and $u\in V$, $cu\in V$
        \item for every $c\in \mathbb{R}$ and $u,v\in V$, $c(u\p v)=cu\p cv$
        \item for every $c,d\in \mathbb{R}$ and $u\in V$, $(c+d)u=cu\p du$
        \item for every $c,d\in \mathbb{R}$ and $u\in V$, $c(du)=(cd)u$
        \item for every $v\in V$, $1v=v$
    \end{enumerate}

    Side note: A space that satisfies 1-5 is called a abelian group.

    \section*{Real Valued Functions as Vector Spaces}
    Let $V$ any function from $X$ to $\mathbb{R}$ where $X$ is any set.
    Define addition on $V$ with inputs $f,g$ in $V$ to be $f(x)+g(x)$.
    Given a $c$ in $\mathbb{R}$ and $f: X \mapsto \mathbb{R}$, define $(cf)$
    to be $cf(x)$.

    \begin{itemize}
        \item Axioms 1 and 6 are immediate.
        \item Axioms 2 and 3 follow from the definition of addition being the
            sum of the outputs of the two functions. The outputs are in
            $\mathbb{R}$ where addition is associative and commutative, thus
            addition is both associative and commutative in this vector space.
        \item Axiom 4: $\vec{0}$ can be defined as the constant function that returns
            zero for all inputs $x$.
            \begin{align*}
                g+0       &= g(x)+f(x) \\
                          &= g(x) + 0 \\
                          &= g(x) \\
                          &= g \\
            \end{align*}
        \item Axiom 5: $-f$ can be defined as the function whose outputs are the
            negation of $f$ for all points $x$
            \begin{align*}
                (-f)   &= -1\cdot f(x) \\
                f+(-f) &= f(x) + -f(x) \\
                       &= 0
            \end{align*}
        \item Axiom 7:
            \begin{align*}
                c(f+g)&=c(f(x)+g(x)) \\
                      &= cf(x) + cg(x) \\
                      &= cf + cg
            \end{align*}
        \item Axiom 8:
            $\mathbb{R}$.
            \begin{align*}
                (c+d)u &= (c+d)u(x) \\
                       &= cu(x)+du(x) \\
                       &= cu+du
            \end{align*}
        \item Axiom 9: The unary function $c$ can be defined as $c\cdot f(x)$.
            \begin{align*}
                c(df) &= c(df(x)) \\
                      &= (cd)f(x) \\
                      &= (cd)f
            \end{align*}
        \item Axiom 10: The scalar 1 is the identity of multiplication in
            $\mathbb{R}$.
            \begin{align*}
                1f &= 1*f(x) \\
                   &= f(x) \\
                   &= f
            \end{align*}
    \end{itemize}

    \section*{Subspaces}
    if $V$ is a vector space and $W$ is a non-empty subset of $V$ then we call
    $W$ a subspace of $V$ if $W$ is closed under addition and scalar
    multiplication. If $W$ is a subspace of $V$, then $W$ itself is a vector
    space.

    \section*{Example 1}
    Prove that $\vec{0}$ is unique.

    Suppose there are two zeros, $\vec{0}, \vec{0}'$.
    \begin{align*}
        \vec{0} + \vec{0}' &= \vec{0} & \text{axiom 4} \\
        \vec{0} + \vec{0}' &= \vec{0}' + \vec{0} & \text{axiom 2} \\
        \vec{0}' + \vec{0} &= \vec{0}' & \text{axiom 4} \\
        \vec{0}       &= \vec{0}'
    \end{align*}

    \section*{Example 2}
    prove that $-(-v) = v$.

    \begin{align*}
        -(-v) = v \\
        0 = v + (-v) \\
        0 = -v + -(-v) \\
        v = -(-v)
    \end{align*}

    \subsection*{Example 3}
    Subspaces of $\mathbb{R}^2$.
    \begin{itemize}
        \item $\mathbb{R}^2$ is a subspace of $\mathbb{R}^2$.
        \item Lines through the origin is a subspace of $\mathbb{R}^2$.
        \item $\vec{0}$ is a subspace of $\mathbb{R}^2$.
    \end{itemize}

    \subsection*{Example 4}
    Let $V_n$ be the set of all polynomials with degree less than or equal to
    $n$. $V_n$ is closed under addition because adding two polynomials doesn't
    increase the degree. Similarly, it is closed in scalar multiplication. This
    proves the $V_n$ is a subspace.
\end{document}
