%%%%%%%%%%%%%%%%%%%%%%%%%%%%% Define Article %%%%%%%%%%%%%%%%%%%%%%%%%%%%%%%%%%
\documentclass{article}
%%%%%%%%%%%%%%%%%%%%%%%%%%%%%%%%%%%%%%%%%%%%%%%%%%%%%%%%%%%%%%%%%%%%%%%%%%%%%%%

%%%%%%%%%%%%%%%%%%%%%%%%%%%%% Using Packages %%%%%%%%%%%%%%%%%%%%%%%%%%%%%%%%%%
\usepackage{geometry}
\usepackage{graphicx}
\usepackage{amssymb}
\usepackage{amsmath}
\usepackage{amsthm}
\usepackage{empheq}
\usepackage{mdframed}
\usepackage{booktabs}
\usepackage{lipsum}
\usepackage{graphicx}
\usepackage{color}
\usepackage{psfrag}
\usepackage{pgfplots}
\usepackage{bm}
%%%%%%%%%%%%%%%%%%%%%%%%%%%%%%%%%%%%%%%%%%%%%%%%%%%%%%%%%%%%%%%%%%%%%%%%%%%%%%%

% Other Settings

%%%%%%%%%%%%%%%%%%%%%%%%%% Page Setting %%%%%%%%%%%%%%%%%%%%%%%%%%%%%%%%%%%%%%%
\geometry{a4paper}

%%%%%%%%%%%%%%%%%%%%%%%%%% Define some useful colors %%%%%%%%%%%%%%%%%%%%%%%%%%
\definecolor{ocre}{RGB}{243,102,25}
\definecolor{mygray}{RGB}{243,243,244}
\definecolor{deepGreen}{RGB}{26,111,0}
\definecolor{shallowGreen}{RGB}{235,255,255}
\definecolor{deepBlue}{RGB}{61,124,222}
\definecolor{shallowBlue}{RGB}{235,249,255}
%%%%%%%%%%%%%%%%%%%%%%%%%%%%%%%%%%%%%%%%%%%%%%%%%%%%%%%%%%%%%%%%%%%%%%%%%%%%%%%

%%%%%%%%%%%%%%%%%%%%%%%%%% Define an orangebox command %%%%%%%%%%%%%%%%%%%%%%%%
\newcommand\orangebox[1]{\fcolorbox{ocre}{mygray}{\hspace{1em}#1\hspace{1em}}}
%%%%%%%%%%%%%%%%%%%%%%%%%%%%%%%%%%%%%%%%%%%%%%%%%%%%%%%%%%%%%%%%%%%%%%%%%%%%%%%

%%%%%%%%%%%%%%%%%%%%%%%%%%%% English Environments %%%%%%%%%%%%%%%%%%%%%%%%%%%%%
\newtheoremstyle{mytheoremstyle}{3pt}{3pt}{\normalfont}{0cm}{\rmfamily\bfseries}{}{1em}{{\color{black}\thmname{#1}~\thmnumber{#2}}\thmnote{\,--\,#3}}
\newtheoremstyle{myproblemstyle}{3pt}{3pt}{\normalfont}{0cm}{\rmfamily\bfseries}{}{1em}{{\color{black}\thmname{#1}~\thmnumber{#2}}\thmnote{\,--\,#3}}
\theoremstyle{mytheoremstyle}
\newmdtheoremenv[linewidth=1pt,backgroundcolor=shallowGreen,linecolor=deepGreen,leftmargin=0pt,innerleftmargin=20pt,innerrightmargin=20pt,]{theorem}{Theorem}[section]
\theoremstyle{mytheoremstyle}
\newmdtheoremenv[linewidth=1pt,backgroundcolor=shallowBlue,linecolor=deepBlue,leftmargin=0pt,innerleftmargin=20pt,innerrightmargin=20pt,]{definition}{Definition}[section]
\theoremstyle{myproblemstyle}
\newmdtheoremenv[linecolor=black,leftmargin=0pt,innerleftmargin=10pt,innerrightmargin=10pt,]{problem}{Problem}[section]
%%%%%%%%%%%%%%%%%%%%%%%%%%%%%%%%%%%%%%%%%%%%%%%%%%%%%%%%%%%%%%%%%%%%%%%%%%%%%%%

%%%%%%%%%%%%%%%%%%%%%%%%%%%%%%% Plotting Settings %%%%%%%%%%%%%%%%%%%%%%%%%%%%%
\usepgfplotslibrary{colorbrewer}
\pgfplotsset{width=8cm,compat=1.9}
%%%%%%%%%%%%%%%%%%%%%%%%%%%%%%%%%%%%%%%%%%%%%%%%%%%%%%%%%%%%%%%%%%%%%%%%%%%%%%%

%%%%%%%%%%%%%%%%%%%%%%%%%%%%%%% Title & Author %%%%%%%%%%%%%%%%%%%%%%%%%%%%%%%%
\title{Dimensions}
\author{Patrick Chen}
\date{Nov 12, 2024}
%%%%%%%%%%%%%%%%%%%%%%%%%%%%%%%%%%%%%%%%%%%%%%%%%%%%%%%%%%%%%%%%%%%%%%%%%%%%%%%

\begin{document}
    \maketitle

    \section*{Dimension}
    Suppose that $V$ has a finite spanning set. Every basis of $V$ is finite and
    has the same size. The size of any such basis is called the dimension of
    $V$, written $\dim(V)$. If $V$ does not have a finite spanning set, then $V$
    is said to be infinite dimensional.

    Suppose $B=\{v_1\dots v_n\}$ is a basis for $V$. Take $w_1\dots w_k$ is a
    linearly independent set in $V$.
    \begin{align*}
        [w_1]_B,\dots ,[w_k]_B \in \mathbb{R}^n
    \end{align*}
    Thus $k$ must be less than or equal to $n$ because $w$ is linearly
    independent.
    \begin{align*}
        c_1[w_1]_B + \dots c_k[w_k]_B = 0
    \end{align*}

    \begin{enumerate}
        \item A set of more than $\dim(V)$ vectors cannot be linearly independent
        \item A set of less than $\dim(V)$ cannot be a spanning set.
        \item Every linearly independent set of size $\dim(V)$ is a basis of $V$
        \item Every spanning set of size $\dim(V)$ is a basis of $V$
        \item Every subspace of $V$ has a dimension less than or equal to $\dim(V)$
    \end{enumerate}

    The dimension of a subspace of a vector space is less than or equal to the
    dimension of the parent space.

    \section*{Coordinates as Linear Transformations}
    Suppose that $B=\{v_1\dots v_n\}$ is a basis for $V$. The map $T: V \mapsto
    \mathbb{R}^n$ given by $T(v)=[v]_B$ is a linear transformation which is
    one-to-one and onto. These types of linear transformations are called
    isomorphisms. Two vector spaces that are isomorphisms are indistinguishable
    in terms of vector spaces. Every basis with the same dimension is isomorphic
    to each other.

    \subsection*{Dimension of different spaces}
    The dimension of the column space is the rank of the matrix. The dimension
    of the nullspace is the nullity of the matrix. The sum of the rank and
    nullity is the number of columns of the matrix.
    \begin{align*}
        A: m\times n \\
        rank(A) + nullity(A) = n
    \end{align*}
    The dimension of the column space is the amount of pivots columns of a
    matrix. The size of the basis of the nullspace is the amount of free
    variables in the reduced row echelon form of the matrix. Since a column in
    the reduced row echelon can only have a pivot or be a free variable, the
    amount of columns is the sum of the rank and nullity of the matrix.
    \begin{align*}
        dim(Col(A)) = dim(Row(A))
    \end{align*}
    Since the column space is the columns that contain the pivots of $A$ and the
    row space has a basis of the non-zero rows in the reduced row echelon form
    of $A$, their dimensions will be equal.

    \subsection*{Example 1}
    Find the dimension of: vectors with $n$ components, polynomials with degree
    less than or equal to $n$, matrices of size $2\times 2$, matrices of size
    $3\times 6$, matrices of size $6\times 3$.
    \begin{enumerate}
        \item $\dim(\mathbb{R}^n)= n$
        \item $\dim(P_n)= n+1$
        \item $\dim(M_{2,2})= 4$
        \item $\dim(M_{3,6})= 18$
        \item $\dim(M_{6,3})= 18$
    \end{enumerate}
\end{document}
