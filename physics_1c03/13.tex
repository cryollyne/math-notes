%%%%%%%%%%%%%%%%%%%%%%%%%%%%% Define Article %%%%%%%%%%%%%%%%%%%%%%%%%%%%%%%%%%
\documentclass{article}
%%%%%%%%%%%%%%%%%%%%%%%%%%%%%%%%%%%%%%%%%%%%%%%%%%%%%%%%%%%%%%%%%%%%%%%%%%%%%%%

%%%%%%%%%%%%%%%%%%%%%%%%%%%%% Using Packages %%%%%%%%%%%%%%%%%%%%%%%%%%%%%%%%%%
\usepackage{geometry}
\usepackage{graphicx}
\usepackage{amssymb}
\usepackage{amsmath}
\usepackage{amsthm}
\usepackage{empheq}
\usepackage{mdframed}
\usepackage{booktabs}
\usepackage{lipsum}
\usepackage{graphicx}
\usepackage{color}
\usepackage{psfrag}
\usepackage{pgfplots}
\usepackage{bm}
%%%%%%%%%%%%%%%%%%%%%%%%%%%%%%%%%%%%%%%%%%%%%%%%%%%%%%%%%%%%%%%%%%%%%%%%%%%%%%%

% Other Settings

%%%%%%%%%%%%%%%%%%%%%%%%%% Page Setting %%%%%%%%%%%%%%%%%%%%%%%%%%%%%%%%%%%%%%%
\geometry{a4paper}

%%%%%%%%%%%%%%%%%%%%%%%%%% Define some useful colors %%%%%%%%%%%%%%%%%%%%%%%%%%
\definecolor{ocre}{RGB}{243,102,25}
\definecolor{mygray}{RGB}{243,243,244}
\definecolor{deepGreen}{RGB}{26,111,0}
\definecolor{shallowGreen}{RGB}{235,255,255}
\definecolor{deepBlue}{RGB}{61,124,222}
\definecolor{shallowBlue}{RGB}{235,249,255}
%%%%%%%%%%%%%%%%%%%%%%%%%%%%%%%%%%%%%%%%%%%%%%%%%%%%%%%%%%%%%%%%%%%%%%%%%%%%%%%

%%%%%%%%%%%%%%%%%%%%%%%%%% Define an orangebox command %%%%%%%%%%%%%%%%%%%%%%%%
\newcommand\orangebox[1]{\fcolorbox{ocre}{mygray}{\hspace{1em}#1\hspace{1em}}}
%%%%%%%%%%%%%%%%%%%%%%%%%%%%%%%%%%%%%%%%%%%%%%%%%%%%%%%%%%%%%%%%%%%%%%%%%%%%%%%

%%%%%%%%%%%%%%%%%%%%%%%%%%%% English Environments %%%%%%%%%%%%%%%%%%%%%%%%%%%%%
\newtheoremstyle{mytheoremstyle}{3pt}{3pt}{\normalfont}{0cm}{\rmfamily\bfseries}{}{1em}{{\color{black}\thmname{#1}~\thmnumber{#2}}\thmnote{\,--\,#3}}
\newtheoremstyle{myproblemstyle}{3pt}{3pt}{\normalfont}{0cm}{\rmfamily\bfseries}{}{1em}{{\color{black}\thmname{#1}~\thmnumber{#2}}\thmnote{\,--\,#3}}
\theoremstyle{mytheoremstyle}
\newmdtheoremenv[linewidth=1pt,backgroundcolor=shallowGreen,linecolor=deepGreen,leftmargin=0pt,innerleftmargin=20pt,innerrightmargin=20pt,]{theorem}{Theorem}[section]
\theoremstyle{mytheoremstyle}
\newmdtheoremenv[linewidth=1pt,backgroundcolor=shallowBlue,linecolor=deepBlue,leftmargin=0pt,innerleftmargin=20pt,innerrightmargin=20pt,]{definition}{Definition}[section]
\theoremstyle{myproblemstyle}
\newmdtheoremenv[linecolor=black,leftmargin=0pt,innerleftmargin=10pt,innerrightmargin=10pt,]{problem}{Problem}[section]
%%%%%%%%%%%%%%%%%%%%%%%%%%%%%%%%%%%%%%%%%%%%%%%%%%%%%%%%%%%%%%%%%%%%%%%%%%%%%%%

%%%%%%%%%%%%%%%%%%%%%%%%%%%%%%% Plotting Settings %%%%%%%%%%%%%%%%%%%%%%%%%%%%%
\usepgfplotslibrary{colorbrewer}
\pgfplotsset{width=8cm,compat=1.9}
%%%%%%%%%%%%%%%%%%%%%%%%%%%%%%%%%%%%%%%%%%%%%%%%%%%%%%%%%%%%%%%%%%%%%%%%%%%%%%%

%%%%%%%%%%%%%%%%%%%%%%%%%%%%%%% Title & Author %%%%%%%%%%%%%%%%%%%%%%%%%%%%%%%%
\title{Simple Harmonic Motion}
\author{Patrick Chen}
\date{Nov 26, 2024}
%%%%%%%%%%%%%%%%%%%%%%%%%%%%%%%%%%%%%%%%%%%%%%%%%%%%%%%%%%%%%%%%%%%%%%%%%%%%%%%

\begin{document}
    \maketitle
    Oscillatory motion is a repetitive back and forth motion about a equilibrium
    position. Periodic motion is common at all length scales. The period ($T$)
    is the time to complete a full cycle. The frequency ($f,\nu$) is the amount
    of times it oscillates per unit time. An example of oscillatory motion is a
    block on a spring.

    \section*{Simple Harmonic Oscillation}
    The spring force is a restoring force that pulls the mass back towards the
    equilibrium point. Simple harmonic motion is a periodic motion. To have
    simple harmonic motion, the restoring force must be proportional to the
    displacement. Simple harmonic motion is always a sinusoidal curve.

    \begin{align*}
        F = -kx \\
        m \frac{d^2 x}{dt^2} = -k x \\
        x(t) = A\cos(\omega t + \Phi)
    \end{align*}
    \begin{itemize}
        \item $A$ is the amplitude
        \item $\omega$ is the angular frequency ($\omega = 2\pi f$)
        \item $\Phi$ is the phase constant
    \end{itemize}

    \subsection*{Angular Frequency}
    The period $T$ is the time needed to complete one cycle.
    \begin{align*}
        x(t) &= A\cos(\omega t+\Phi) \\
        &= A\cos(\omega t+\Phi + 2\pi) \\
        x(t+T) &= x(t)
    \end{align*}
    thus $\omega t = 2\pi$
    \begin{align*}
        \omega = \frac{2\pi}{T} = 2\pi f
    \end{align*}
    The angular frequency is measured in radians per second.

    \subsection*{Phase}
    The initial phase determines where the in motion we set $t=0$.

    \section*{Velocity and Acceleration}
    \begin{align*}
        x(t) &= A\cos(\omega t + \Phi) \\
        v(t) &= -A\omega\sin(\omega t + \Phi) \\
        a(t) &= -A\omega^2\cos(\omega t + \Phi) = -\omega^2x(t) \\
        v_{max} &= A\omega \\
        a_{max} &= A\omega^2
    \end{align*}
    The maximum velocity occurs when the position is at the equilibrium point.
    The maximum acceleration occurs when the position is farthest away from the
    equilibrium point. The phase and amplitude can be computed from the starting
    positions and velocities.

    \begin{align*}
        t   &= 0 \\
        x_0 &= A\cos(\omega t + \Phi) \\
            &= A\cos(\Phi) \\
        v_0 &= -A\omega\sin(\omega t + \Phi) \\
            &= -A\omega\sin(\Phi) \\
    \end{align*}

    \begin{align*}
        \frac{v_0}{x_0} &= -\frac{A\omega\sin(\Phi)}{A\cos(\Phi)} \\
        \frac{v_0}{x_0} &= -\omega\tan(\Phi) \\
        \tan^{-1}(-\frac{v_0}{\omega x_0}) &= \Phi
    \end{align*}

    \begin{align*}
        x_0^2+(\frac{v_0}{\omega})^2 &= (-A\sin(\Phi))^2 + (A\cos(\Phi))^2 \\
        x_0^2+(\frac{v_0}{\omega})^2 &= A^2\sin^2(\Phi) + A^2\cos^2(\Phi) \\
        x_0^2+(\frac{v_0}{\omega})^2 &= A^2(\sin^2(\Phi) + \cos^2(\Phi)) \\
        x_0^2+(\frac{v_0}{\omega})^2 &= A^2 \\
        \sqrt{x_0^2+(\frac{v_0}{\omega})^2} &= A
    \end{align*}

    \begin{align*}
        \Phi = \tan^{-1}(\frac{v_0}{\omega  x_0}) \\
        A = \sqrt{x_0^2+ (\frac{v_0}{\omega})^2}
    \end{align*}

    \subsection*{Spring}

    \begin{align*}
        F &= -kx = ma \\
        a &= \frac{d^2 x}{dt^2} = -\frac{k}{m} x \\
        a &= -\omega^2 x \\
        \omega^2 &= \frac{k}{m} \\
        \omega &= \sqrt{\frac{k}{m}}
    \end{align*}
    The frequency of the oscillations depends on only the material properties of
    the object and not the amplitude

    \begin{align*}
        T = \frac{2\pi}{\omega} = \frac{2\pi}{\sqrt{\frac{k}{m}}} \\
        f = \frac{\omega}{2\pi} = \frac{\sqrt{\frac{k}{m}}}{2\pi}
    \end{align*}

    \section*{Energy}
    The oscillator will have both kinetic and spring potential. There are no
    non-conservative forces in the system.
    \begin{align*}
        K &= \frac{1}{2} mv^2 \\
        &= \frac{1}{2} m \omega^2 A^2 \sin^2(\omega t+ \Phi) \\
        U &= \frac{1}{2} kx^2 \\
        &= \frac{1}{2} kA^2 \cos^2(\omega t + \Phi) \\
        K+U &= \frac{1}{2} A^2 (m\omega^2\sin^2(\omega t + \Phi) + k \cos^2(\omega t + \Phi)) \\
        &= \frac{1}{2} A^2 (k\sin^2(\omega t + \Phi) + k \cos^2(\omega t + \Phi)) \\
        &= \frac{1}{2} kA^2 (\sin^2(\omega t + \Phi) + \cos^2(\omega t + \Phi)) \\
        &= \frac{1}{2} kA^2 \\
    \end{align*}
    This could also be written as:
    \begin{align*}
        K+U &= \frac{1}{2} mv_{max}^2
    \end{align*}

    Using these formulas, the max velocity and velocity function with respect to
    position can be calculated.
    \begin{align*}
        E = \frac{1}{2} mv_{max}^2 &= \frac{1}{2} kA^2 \\
        v_{max}^2 &= \frac{k}{m} A^2 \\
        v_{max}^2 &= \omega^2 A^2 \\
        v_{max} &= \sqrt{\omega^2 A^2} \\
        v_{max} &= \omega A
    \end{align*}

    \begin{align*}
        \frac{1}{2} kx^2 + \frac{1}{2} mv^2 &= \frac{1}{2} kA^2 \\
        kx^2 + mv^2 &= kA^2 \\
        mv^2 &= kA^2 - kx^2 \\
        v^2 &= \frac{k}{m} (A^2 - x^2) \\
        v &= \sqrt{\omega^2 (A^2 - x^2)} \\
        v &= \omega\sqrt{A^2 - x^2}
    \end{align*}

    \begin{align*}
        v_{max} &= \omega A \\
        v &= \omega \sqrt{A^2-x^2}
    \end{align*}

    \subsection*{Example 1}
    A 0.25 kg block is oscillating on a spring with $k=4$ N/m. At $t=0$ s,
    $V=-0.2$ m/s and $a=+0.5 m/s^2$. Find its total energy and equation of
    motion.

    \begin{align*}
        v = -A\omega\sin(\omega t + \Phi) \\
        -0.2 = -A\sqrt{\frac{k}{m}}\sin(\Phi) \\
        -0.2 = -A\sqrt{\frac{4}{0.25}}\sin(\Phi) \\
        -0.2 = -4A\sin(\Phi) \\
    \end{align*}
    \begin{align*}
        a = -A\omega^2\cos(\omega t + \Phi) \\
        0.5 = -\frac{k}{m} A\cos(\Phi) \\
        0.5 = -\frac{4}{0.25} A\cos(\Phi) \\
        0.5 = -16 A\cos(\Phi)
    \end{align*}
    Divide velocity equation by acceleration equation:

    \begin{align*}
        -\frac{0.2}{0.5} = \frac{-4A\sin(\Phi)}{-16A\cos(\Phi)} \\
        -\frac{0.2}{0.5} = \frac{\sin(\Phi)}{4\cos(\Phi)} \\
        -1.6 = \tan(\Phi) \\
        \Phi = 2.12 \text{ or } -1.01
    \end{align*}
    Using the sign from the velocity and acceleration, we can determine that
    $\Phi=2.12$.

    \begin{align*}
        -0.2 &= -4A\sin(2.12) \\
        A &= 0.059 m \\
    \end{align*}
    \begin{align*}
        E &= \frac{1}{2} kA^2 \\
        &= \frac{1}{2} (4)(0.059)^2 \\
        &= 0.007 J
    \end{align*}

    \begin{align*}
        x(t) &= 0.059\cos(4t + 2.12)
        E &= 0.007 J
    \end{align*}

\end{document}
