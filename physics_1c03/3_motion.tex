%%%%%%%%%%%%%%%%%%%%%%%%%%%%% Define Article %%%%%%%%%%%%%%%%%%%%%%%%%%%%%%%%%%
\documentclass{article}
%%%%%%%%%%%%%%%%%%%%%%%%%%%%%%%%%%%%%%%%%%%%%%%%%%%%%%%%%%%%%%%%%%%%%%%%%%%%%%%

%%%%%%%%%%%%%%%%%%%%%%%%%%%%% Using Packages %%%%%%%%%%%%%%%%%%%%%%%%%%%%%%%%%%
\usepackage{geometry}
\usepackage{graphicx}
\usepackage{amssymb}
\usepackage{amsmath}
\usepackage{amsthm}
\usepackage{empheq}
\usepackage{mdframed}
\usepackage{booktabs}
\usepackage{lipsum}
\usepackage{graphicx}
\usepackage{color}
\usepackage{psfrag}
\usepackage{pgfplots}
\usepackage{bm}
%%%%%%%%%%%%%%%%%%%%%%%%%%%%%%%%%%%%%%%%%%%%%%%%%%%%%%%%%%%%%%%%%%%%%%%%%%%%%%%

% Other Settings

%%%%%%%%%%%%%%%%%%%%%%%%%% Page Setting %%%%%%%%%%%%%%%%%%%%%%%%%%%%%%%%%%%%%%%
\geometry{a4paper}

%%%%%%%%%%%%%%%%%%%%%%%%%% Define some useful colors %%%%%%%%%%%%%%%%%%%%%%%%%%
\definecolor{ocre}{RGB}{243,102,25}
\definecolor{mygray}{RGB}{243,243,244}
\definecolor{deepGreen}{RGB}{26,111,0}
\definecolor{shallowGreen}{RGB}{235,255,255}
\definecolor{deepBlue}{RGB}{61,124,222}
\definecolor{shallowBlue}{RGB}{235,249,255}
%%%%%%%%%%%%%%%%%%%%%%%%%%%%%%%%%%%%%%%%%%%%%%%%%%%%%%%%%%%%%%%%%%%%%%%%%%%%%%%

%%%%%%%%%%%%%%%%%%%%%%%%%% Define an orangebox command %%%%%%%%%%%%%%%%%%%%%%%%
\newcommand\orangebox[1]{\fcolorbox{ocre}{mygray}{\hspace{1em}#1\hspace{1em}}}
%%%%%%%%%%%%%%%%%%%%%%%%%%%%%%%%%%%%%%%%%%%%%%%%%%%%%%%%%%%%%%%%%%%%%%%%%%%%%%%

%%%%%%%%%%%%%%%%%%%%%%%%%%%% English Environments %%%%%%%%%%%%%%%%%%%%%%%%%%%%%
\newtheoremstyle{mytheoremstyle}{3pt}{3pt}{\normalfont}{0cm}{\rmfamily\bfseries}{}{1em}{{\color{black}\thmname{#1}~\thmnumber{#2}}\thmnote{\,--\,#3}}
\newtheoremstyle{myproblemstyle}{3pt}{3pt}{\normalfont}{0cm}{\rmfamily\bfseries}{}{1em}{{\color{black}\thmname{#1}~\thmnumber{#2}}\thmnote{\,--\,#3}}
\theoremstyle{mytheoremstyle}
\newmdtheoremenv[linewidth=1pt,backgroundcolor=shallowGreen,linecolor=deepGreen,leftmargin=0pt,innerleftmargin=20pt,innerrightmargin=20pt,]{theorem}{Theorem}[section]
\theoremstyle{mytheoremstyle}
\newmdtheoremenv[linewidth=1pt,backgroundcolor=shallowBlue,linecolor=deepBlue,leftmargin=0pt,innerleftmargin=20pt,innerrightmargin=20pt,]{definition}{Definition}[section]
\theoremstyle{myproblemstyle}
\newmdtheoremenv[linecolor=black,leftmargin=0pt,innerleftmargin=10pt,innerrightmargin=10pt,]{problem}{Problem}[section]
%%%%%%%%%%%%%%%%%%%%%%%%%%%%%%%%%%%%%%%%%%%%%%%%%%%%%%%%%%%%%%%%%%%%%%%%%%%%%%%

%%%%%%%%%%%%%%%%%%%%%%%%%%%%%%% Plotting Settings %%%%%%%%%%%%%%%%%%%%%%%%%%%%%
\usepgfplotslibrary{colorbrewer}
\pgfplotsset{width=8cm,compat=1.9}
%%%%%%%%%%%%%%%%%%%%%%%%%%%%%%%%%%%%%%%%%%%%%%%%%%%%%%%%%%%%%%%%%%%%%%%%%%%%%%%

%%%%%%%%%%%%%%%%%%%%%%%%%%%%%%% Title & Author %%%%%%%%%%%%%%%%%%%%%%%%%%%%%%%%
\title{Motion}
\author{Patrick Chen}
\date{Sept 12, 2024}
%%%%%%%%%%%%%%%%%%%%%%%%%%%%%%%%%%%%%%%%%%%%%%%%%%%%%%%%%%%%%%%%%%%%%%%%%%%%%%%

\begin{document}
    \maketitle
    \section*{Free fall}
    Motion that is only affected by gravity is called free fall. All objects in
    free fall move with a constant downward acceleration.
    \begin{equation*}
        a = g \approx 9.8 \frac{m}{s^2}
    \end{equation*}
    The acceleration is the same for all objects, but the acceleration may
    differ depending on where you are.
    \begin{align*}
        v= v_0 + at &\rightarrow b = v_0 - gt \\
        x = x_0 + vt + \frac{1}{2} at &\rightarrow x = x_0 + vt - \frac{1}{2} gt
    \end{align*}

    Assuming up is positive,
    \begin{itemize}
        \item The position of the thrown object is concave down parabola
            \begin{align*}
                y = y_0 +v_{0y} + \frac{1}{2} at^2
            \end{align*}
        \item The velocity of the thrown object is linear with a negative slope
            \begin{align*}
                v_y = v_{0y} + at
            \end{align*}
        \item The acceleration of a thrown object is a negative constant.
            \begin{align*}
                a_y = a
            \end{align*}
    \end{itemize}

    At the top of the arc of the thrown object, the velocity is $0 \frac{m}{s}$
    and the acceleration is $-9.8 \frac{m}{s^2}$.

    \subsection*{Strategies for solving questions}
    \begin{itemize}
        \item If there are separate types of motions, you can break it up into
            different parts.
        \item Strategically choose where the origin is.
    \end{itemize}

    \subsection*{Example}
    \begin{align*}
        v_0 = 15 m/s && x_i = 2m && x_f = 0m \\
    \end{align*}
    \begin{align*}
        x_f = x_i + v_0t + \frac{1}{2} at^2 \\
        0 = 2 + 15t + \frac{1}{2} (-9.8) t^2 \\
        0 = -4.9 t^2 + 15t + 2 \\
    \end{align*}
    \begin{align*}
        t \approx 3.189207165165621 s \text{ or } t \approx -0.1279826753697030 s
    \end{align*}
    Since t cannot be negative
    \begin{align*}
        t \approx 3.189207165165621 s
    \end{align*}
    \begin{align*}
        v &= v_0 + at \\
          &= 15 + (-9.8)\cdot 3.189207165165621 \\
          &= -16.25423021862309 \frac{m}{s}
    \end{align*}
\end{document}
