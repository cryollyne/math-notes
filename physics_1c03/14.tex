%%%%%%%%%%%%%%%%%%%%%%%%%%%%% Define Article %%%%%%%%%%%%%%%%%%%%%%%%%%%%%%%%%%
\documentclass{article}
%%%%%%%%%%%%%%%%%%%%%%%%%%%%%%%%%%%%%%%%%%%%%%%%%%%%%%%%%%%%%%%%%%%%%%%%%%%%%%%

%%%%%%%%%%%%%%%%%%%%%%%%%%%%% Using Packages %%%%%%%%%%%%%%%%%%%%%%%%%%%%%%%%%%
\usepackage{geometry}
\usepackage{graphicx}
\usepackage{amssymb}
\usepackage{amsmath}
\usepackage{amsthm}
\usepackage{empheq}
\usepackage{mdframed}
\usepackage{booktabs}
\usepackage{lipsum}
\usepackage{graphicx}
\usepackage{color}
\usepackage{psfrag}
\usepackage{pgfplots}
\usepackage{bm}
%%%%%%%%%%%%%%%%%%%%%%%%%%%%%%%%%%%%%%%%%%%%%%%%%%%%%%%%%%%%%%%%%%%%%%%%%%%%%%%

% Other Settings

%%%%%%%%%%%%%%%%%%%%%%%%%% Page Setting %%%%%%%%%%%%%%%%%%%%%%%%%%%%%%%%%%%%%%%
\geometry{a4paper}

%%%%%%%%%%%%%%%%%%%%%%%%%% Define some useful colors %%%%%%%%%%%%%%%%%%%%%%%%%%
\definecolor{ocre}{RGB}{243,102,25}
\definecolor{mygray}{RGB}{243,243,244}
\definecolor{deepGreen}{RGB}{26,111,0}
\definecolor{shallowGreen}{RGB}{235,255,255}
\definecolor{deepBlue}{RGB}{61,124,222}
\definecolor{shallowBlue}{RGB}{235,249,255}
%%%%%%%%%%%%%%%%%%%%%%%%%%%%%%%%%%%%%%%%%%%%%%%%%%%%%%%%%%%%%%%%%%%%%%%%%%%%%%%

%%%%%%%%%%%%%%%%%%%%%%%%%% Define an orangebox command %%%%%%%%%%%%%%%%%%%%%%%%
\newcommand\orangebox[1]{\fcolorbox{ocre}{mygray}{\hspace{1em}#1\hspace{1em}}}
%%%%%%%%%%%%%%%%%%%%%%%%%%%%%%%%%%%%%%%%%%%%%%%%%%%%%%%%%%%%%%%%%%%%%%%%%%%%%%%

%%%%%%%%%%%%%%%%%%%%%%%%%%%% English Environments %%%%%%%%%%%%%%%%%%%%%%%%%%%%%
\newtheoremstyle{mytheoremstyle}{3pt}{3pt}{\normalfont}{0cm}{\rmfamily\bfseries}{}{1em}{{\color{black}\thmname{#1}~\thmnumber{#2}}\thmnote{\,--\,#3}}
\newtheoremstyle{myproblemstyle}{3pt}{3pt}{\normalfont}{0cm}{\rmfamily\bfseries}{}{1em}{{\color{black}\thmname{#1}~\thmnumber{#2}}\thmnote{\,--\,#3}}
\theoremstyle{mytheoremstyle}
\newmdtheoremenv[linewidth=1pt,backgroundcolor=shallowGreen,linecolor=deepGreen,leftmargin=0pt,innerleftmargin=20pt,innerrightmargin=20pt,]{theorem}{Theorem}[section]
\theoremstyle{mytheoremstyle}
\newmdtheoremenv[linewidth=1pt,backgroundcolor=shallowBlue,linecolor=deepBlue,leftmargin=0pt,innerleftmargin=20pt,innerrightmargin=20pt,]{definition}{Definition}[section]
\theoremstyle{myproblemstyle}
\newmdtheoremenv[linecolor=black,leftmargin=0pt,innerleftmargin=10pt,innerrightmargin=10pt,]{problem}{Problem}[section]
%%%%%%%%%%%%%%%%%%%%%%%%%%%%%%%%%%%%%%%%%%%%%%%%%%%%%%%%%%%%%%%%%%%%%%%%%%%%%%%

%%%%%%%%%%%%%%%%%%%%%%%%%%%%%%% Plotting Settings %%%%%%%%%%%%%%%%%%%%%%%%%%%%%
\usepgfplotslibrary{colorbrewer}
\pgfplotsset{width=8cm,compat=1.9}
%%%%%%%%%%%%%%%%%%%%%%%%%%%%%%%%%%%%%%%%%%%%%%%%%%%%%%%%%%%%%%%%%%%%%%%%%%%%%%%

%%%%%%%%%%%%%%%%%%%%%%%%%%%%%%% Title & Author %%%%%%%%%%%%%%%%%%%%%%%%%%%%%%%%
\title{Fluids}
\author{Patrick Chen}
\date{Dec 3, 2024}
%%%%%%%%%%%%%%%%%%%%%%%%%%%%%%%%%%%%%%%%%%%%%%%%%%%%%%%%%%%%%%%%%%%%%%%%%%%%%%%

\begin{document}
    \maketitle
    Solids resists deformations and maintains shape and volume. A solid material
    is rigid. Fluids flow and are easily deformed in response to an applied
    force. They take the shape of the container. Liquids are nearly
    incompressible and gasses are compressible.

    \section*{Pressure}
    Pressure is the amount of force per unit area. Gasses have pressure because
    of the collisions of the molecules with the walls of the container. In the
    SI system, pressure is measured in $N/m^2$ or $Pa$ (Pascal). Alternatively,
    one atmosphere of pressure is the atmospheric pressure at mean see level at
    the latitude of Paris, France. $1\ atm = 1.013\times 10^5 Pa$. The imperial
    unit of pressure is the psi (pounds per square inch). $1 atm = 14.7 psi$.

    \subsection*{Hydrostatic Pressure}
    Pressure increases deeper in a liquid. Consider a cylinder of water in a
    swimming pool of volume $V=Ah$.
    \begin{align*}
        P_{top} \\
        P_{bot} &= P_{top} + \Delta P \\
        \sum F = 0 &= P_{bot}A - (mg + P_{top}A) \\
        0 &= (P_{top}+\Delta P) A - mg - P_{top}A \\
        0 &= \Delta P A - mg \\
        \frac{mg}{A} &= \Delta P \\
    \end{align*}
    since $m = \rho V = \rho A h$ where $\rho$ is the density:
    \begin{align*}
        \Delta p &= \rho g h
    \end{align*}

    \subsection*{Pascal's Principle}
    A pressure change applied to an enclosed liquid is transmitted undiminished
    to every part of the liquid and to the wall of the container in contact with
    the liquid. This means that two points at the same height will experience
    the same pressure.

    \section*{Absolute Pressure}
    A perfect vacuum with no molecules to collide with the walls of the
    container is 0 Pa. This is the "absolute zero" of pressure because it is not
    possible to have a lower pressure.

    \subsection*{Gauge Pressure}
    Gauge pressure is the difference in pressure compared to some reference
    point (often the atmospheric pressure $P_1 = 1$ atm). Since it is a relative
    measurement, it can be either positive or negative.
    \begin{align*}
        P_{gauge} = \Delta P = P_2 - P_1
    \end{align*}

    \subsection*{Example 1}
    When drinking through a straw, we can reduce the pressure in our lungs by
    -85 mmHg. Suppose the drink has a density of $0.92\times 10^3\ kg/m^3$ What
    is the maximum height of straw that we could use.

    \begin{align*}
        \Delta P &= -85\ mmHg \\
        \Delta P &= -1.13\times 10^4\ Pa \\
        \rho gh &= \Delta P \\
        (0.92) (9.8)h &= -1.13\times 10^4 \\
        h &= -1.25 \\
        d &= 1.25\text{ m}
    \end{align*}

    \subsection*{Buoyancy}
    A block suspended in water will have a buoyancy force acting on it.
    \begin{align*}
        F &= -\rho g V \\
          &= -\rho g A x \\
        F &\propto -x \\
        k &= \rho g A
    \end{align*}
    Since the force is proportional to the negative displacement, it is simple
    harmonic motion.

    \begin{align*}
        \omega &= \sqrt{\frac{k}{m}} \\
        &= \sqrt{\frac{\rho g A}{\rho V}} \\
        &= \sqrt{\frac{g A}{V}}
    \end{align*}

    \section*{Continuity Equation}
    For fluid flowing in a tube, mass must be conserved. This means that for
    incompressible fluids, the volume flowing into a tube must be the same as
    the volume flowing out of the tube. This means that the velocity times the
    cross sectional area is constant across
    the entire tube.
    \begin{align*}
        V_1 &= V_2 \\
        \Delta x_1 A_1 &= \Delta x_2 A_2 \\
        v_1 A_1 &= v_2 A_2
    \end{align*}

    The volume flow rate ($Q$) can be defined as:
    \begin{align*}
        Q = Av
    \end{align*}

    \subsection*{Bernoulli's principle}
    Suppose there is pressure on both sides of a tube acting on the same volume.
    \begin{align*}
        W_{tot} &= \Delta K + \Delta U \\
        P_1A_1\Delta x_1 - P_2A_2\Delta x_2
                &= (\frac{1}{2} m_2v_2^2 - \frac{1}{2} m_1v_1^2)
                + (m_2gy_2 - m_1gy_1) \\
        P_1V-P_2V
                &= \frac{1}{2} (\rho V)v_2^2 - \frac{1}{2} (\rho V)v_1^2
                + (\rho V)gy_2 - (\rho V)gy_1 \\
        P_1-P_2 &= \frac{1}{2} \rho v_2^2 - \frac{1}{2} \rho v_1^2
                + \rho g y_2 - \rho g y_1\\
        P_1 + \rho g y_1 + \frac{1}{2} \rho v_1^2
                &= P_2 + \rho g y_2 + \frac{1}{2} \rho v_2^2
    \end{align*}
    This is known as Bernoulli's principle. $P + \rho gy + \frac{1}{2} \rho v^2$
    is constant throughout the fluid.

\end{document}
