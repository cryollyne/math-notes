%%%%%%%%%%%%%%%%%%%%%%%%%%%%% Define Article %%%%%%%%%%%%%%%%%%%%%%%%%%%%%%%%%%
\documentclass{article}
%%%%%%%%%%%%%%%%%%%%%%%%%%%%%%%%%%%%%%%%%%%%%%%%%%%%%%%%%%%%%%%%%%%%%%%%%%%%%%%

%%%%%%%%%%%%%%%%%%%%%%%%%%%%% Using Packages %%%%%%%%%%%%%%%%%%%%%%%%%%%%%%%%%%
\usepackage{geometry}
\usepackage{graphicx}
\usepackage{amssymb}
\usepackage{amsmath}
\usepackage{amsthm}
\usepackage{empheq}
\usepackage{mdframed}
\usepackage{booktabs}
\usepackage{lipsum}
\usepackage{graphicx}
\usepackage{color}
\usepackage{psfrag}
\usepackage{pgfplots}
\usepackage{bm}
%%%%%%%%%%%%%%%%%%%%%%%%%%%%%%%%%%%%%%%%%%%%%%%%%%%%%%%%%%%%%%%%%%%%%%%%%%%%%%%

% Other Settings

%%%%%%%%%%%%%%%%%%%%%%%%%% Page Setting %%%%%%%%%%%%%%%%%%%%%%%%%%%%%%%%%%%%%%%
\geometry{a4paper}

%%%%%%%%%%%%%%%%%%%%%%%%%% Define some useful colors %%%%%%%%%%%%%%%%%%%%%%%%%%
\definecolor{ocre}{RGB}{243,102,25}
\definecolor{mygray}{RGB}{243,243,244}
\definecolor{deepGreen}{RGB}{26,111,0}
\definecolor{shallowGreen}{RGB}{235,255,255}
\definecolor{deepBlue}{RGB}{61,124,222}
\definecolor{shallowBlue}{RGB}{235,249,255}
%%%%%%%%%%%%%%%%%%%%%%%%%%%%%%%%%%%%%%%%%%%%%%%%%%%%%%%%%%%%%%%%%%%%%%%%%%%%%%%

%%%%%%%%%%%%%%%%%%%%%%%%%% Define an orangebox command %%%%%%%%%%%%%%%%%%%%%%%%
\newcommand\orangebox[1]{\fcolorbox{ocre}{mygray}{\hspace{1em}#1\hspace{1em}}}
%%%%%%%%%%%%%%%%%%%%%%%%%%%%%%%%%%%%%%%%%%%%%%%%%%%%%%%%%%%%%%%%%%%%%%%%%%%%%%%

%%%%%%%%%%%%%%%%%%%%%%%%%%%% English Environments %%%%%%%%%%%%%%%%%%%%%%%%%%%%%
\newtheoremstyle{mytheoremstyle}{3pt}{3pt}{\normalfont}{0cm}{\rmfamily\bfseries}{}{1em}{{\color{black}\thmname{#1}~\thmnumber{#2}}\thmnote{\,--\,#3}}
\newtheoremstyle{myproblemstyle}{3pt}{3pt}{\normalfont}{0cm}{\rmfamily\bfseries}{}{1em}{{\color{black}\thmname{#1}~\thmnumber{#2}}\thmnote{\,--\,#3}}
\theoremstyle{mytheoremstyle}
\newmdtheoremenv[linewidth=1pt,backgroundcolor=shallowGreen,linecolor=deepGreen,leftmargin=0pt,innerleftmargin=20pt,innerrightmargin=20pt,]{theorem}{Theorem}[section]
\theoremstyle{mytheoremstyle}
\newmdtheoremenv[linewidth=1pt,backgroundcolor=shallowBlue,linecolor=deepBlue,leftmargin=0pt,innerleftmargin=20pt,innerrightmargin=20pt,]{definition}{Definition}[section]
\theoremstyle{myproblemstyle}
\newmdtheoremenv[linecolor=black,leftmargin=0pt,innerleftmargin=10pt,innerrightmargin=10pt,]{problem}{Problem}[section]
%%%%%%%%%%%%%%%%%%%%%%%%%%%%%%%%%%%%%%%%%%%%%%%%%%%%%%%%%%%%%%%%%%%%%%%%%%%%%%%

%%%%%%%%%%%%%%%%%%%%%%%%%%%%%%% Plotting Settings %%%%%%%%%%%%%%%%%%%%%%%%%%%%%
\usepgfplotslibrary{colorbrewer}
\pgfplotsset{width=8cm,compat=1.9}
%%%%%%%%%%%%%%%%%%%%%%%%%%%%%%%%%%%%%%%%%%%%%%%%%%%%%%%%%%%%%%%%%%%%%%%%%%%%%%%

%%%%%%%%%%%%%%%%%%%%%%%%%%%%%%% Title & Author %%%%%%%%%%%%%%%%%%%%%%%%%%%%%%%%
\title{Kinematics in 1D}
\author{Patrick Chen}
\date{Sept 5, 2024}
%%%%%%%%%%%%%%%%%%%%%%%%%%%%%%%%%%%%%%%%%%%%%%%%%%%%%%%%%%%%%%%%%%%%%%%%%%%%%%%

\begin{document}
    \maketitle
    \section*{Dimensional Analysis}
    A dimension (unit) is an inherent property of a quantity. Most physical
    quantities can be expressed in terms of these fundamental dimensions: Mass,
    Length, Time. We can use dimensions to understand and solve problems.

    \subsection*{Rules}
    \begin{enumerate}
        \item addition and subtraction must have same dimensions.

        \item  log and exponential functions only apply to dimensionless quantities

        \item trig functions only apply to dimensionless quantities
    \end{enumerate}

    \subsection*{Example}
    In the equation $E=mc^2$, what is the unit of $E$?
    \begin{align*}
        E& =mc^2 \\
         & = mass\ speed^2 \\
         & = mass\ (\frac{length}{time})^2 \\
         & = mass\ length^2\ time^{-2} \\
         & = energy
    \end{align*}

    Consider the following equation
    \begin{align*}
        x(=?)x_0+v_0t^2+at
    \end{align*}
    The first term has unit of $length$ and the second term has units of
    $length\ time$. These two terms are different units and cannot be added
    therefore this equation is invalid.

    \begin{align*}
        x&=x_0+v_0t^2+at \\
        &= length + length/time \ time^2 + length/time^2\ time \\
        &= length + \frac{length}{time} \ time^2 + \frac{length}{time^2}\ time \\
        &= length + length\ time + \frac{length}{time}
    \end{align*}

    The correct formula is
    \begin{align*}
        x = x_0 + v_0 t + \frac{1}{2} a t^2
    \end{align*}

    \section*{Kinematics}
    Kinematics is the description of motion. These three important quantities
    can describe 1D motion as a function of time:
    \begin{itemize}
        \item Position: $\vec{r}(t)$ (displacement from the origin)
        \item Velocity: $\vec{v}(t)$ (rate of change of position)
        \item Acceleration $\vec{a}(t)$ (rate of change of velocity)
    \end{itemize}

    In 1D direction can be expressed as sign. The relations between these
    quantities and be found using calculus.

    \subsection*{Vector vs Scalar}
    Physical quantities can be classified as scalars, vectors, etc. \\
    Scalars are described as real numbers with units (examples: speed,
    temperature, mass, time, energy) \\
    Vectors describe both a scalar (magnitude) and a direction in space
    (examples: velocity displacement, electric field, force)

    \subsection*{Notation}
    \begin{itemize}
        \item scalars: ordinary or italic font ($i$, $\textit{i}$)

        \item vector: boldface, arrow notation, underline ($\textbf{i}$, $\vec{i}$, $\underline{i}$)

        \item unit vector: hat ($\hat{i}$ or $\hat{x}$ for x, $\hat{j}$ or
            $\hat{y}$ for y, $\hat{k}$ or $\hat{z}$ for z)
    \end{itemize}

    Unit vectors are vectors with a magnitude of $1$. It is basically a
    direction.

    \subsection*{Position, distance, displacement}
    \begin{itemize}
        \item Position: gives a location, need to define an origin

        \item Distance: a scalar that expressed the change in position,
            depending on the path

        \item Displacement: a vector that expressed the straight line distance
            without depending on the path
    \end{itemize}

    Displacement:
    \begin{equation*}
        \Delta \vec{x} \equiv \vec{x}_f - \vec{x}_i
    \end{equation*}

    \subsection*{Vector arithmetic}
    \begin{itemize}
        \item magnitude: scalar, length of vector

        \item scalar * vector: multiply the magnitude of vector

        \item vector * vector: either dot or cross product

        \item vector + vector: add vectors tip to tail

        \item vector - vector: add the negation of the subtracted vector
    \end{itemize}

    \subsection*{Velocity}
    Velocity is the time rate of change of position. Average velocity:
    \begin{align*}
        \vec{v}_{avg}=\frac{\Delta \vec{x}}{\Delta t}
    \end{align*}

    The instantaneous velocity is the slope of a line tangent to curve. Thus,
    velocity is the time derivative of position.
    \begin{align*}
        \vec{v}=\frac{d\vec{x}}{dt}
    \end{align*}

    \subsection*{Acceleration}
    Acceleration is the time rate of change of velocity. Average acceleration:
    \begin{align*}
        \vec{a}_{avg}=\frac{\Delta \vec{v}}{\Delta t}
    \end{align*}

    The instantaneous acceleration is the slope of a line tangent to curve.
    \begin{align*}
        \vec{a}=\frac{d\vec{v}}{dt}
    \end{align*}


\end{document}
