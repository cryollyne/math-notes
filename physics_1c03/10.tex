%%%%%%%%%%%%%%%%%%%%%%%%%%%%% Define Article %%%%%%%%%%%%%%%%%%%%%%%%%%%%%%%%%%
\documentclass{article}
%%%%%%%%%%%%%%%%%%%%%%%%%%%%%%%%%%%%%%%%%%%%%%%%%%%%%%%%%%%%%%%%%%%%%%%%%%%%%%%

%%%%%%%%%%%%%%%%%%%%%%%%%%%%% Using Packages %%%%%%%%%%%%%%%%%%%%%%%%%%%%%%%%%%
\usepackage{geometry}
\usepackage{graphicx}
\usepackage{amssymb}
\usepackage{amsmath}
\usepackage{amsthm}
\usepackage{empheq}
\usepackage{mdframed}
\usepackage{booktabs}
\usepackage{lipsum}
\usepackage{graphicx}
\usepackage{color}
\usepackage{psfrag}
\usepackage{pgfplots}
\usepackage{bm}
%%%%%%%%%%%%%%%%%%%%%%%%%%%%%%%%%%%%%%%%%%%%%%%%%%%%%%%%%%%%%%%%%%%%%%%%%%%%%%%

% Other Settings

%%%%%%%%%%%%%%%%%%%%%%%%%% Page Setting %%%%%%%%%%%%%%%%%%%%%%%%%%%%%%%%%%%%%%%
\geometry{a4paper}

%%%%%%%%%%%%%%%%%%%%%%%%%% Define some useful colors %%%%%%%%%%%%%%%%%%%%%%%%%%
\definecolor{ocre}{RGB}{243,102,25}
\definecolor{mygray}{RGB}{243,243,244}
\definecolor{deepGreen}{RGB}{26,111,0}
\definecolor{shallowGreen}{RGB}{235,255,255}
\definecolor{deepBlue}{RGB}{61,124,222}
\definecolor{shallowBlue}{RGB}{235,249,255}
%%%%%%%%%%%%%%%%%%%%%%%%%%%%%%%%%%%%%%%%%%%%%%%%%%%%%%%%%%%%%%%%%%%%%%%%%%%%%%%

%%%%%%%%%%%%%%%%%%%%%%%%%% Define an orangebox command %%%%%%%%%%%%%%%%%%%%%%%%
\newcommand\orangebox[1]{\fcolorbox{ocre}{mygray}{\hspace{1em}#1\hspace{1em}}}
%%%%%%%%%%%%%%%%%%%%%%%%%%%%%%%%%%%%%%%%%%%%%%%%%%%%%%%%%%%%%%%%%%%%%%%%%%%%%%%

%%%%%%%%%%%%%%%%%%%%%%%%%%%% English Environments %%%%%%%%%%%%%%%%%%%%%%%%%%%%%
\newtheoremstyle{mytheoremstyle}{3pt}{3pt}{\normalfont}{0cm}{\rmfamily\bfseries}{}{1em}{{\color{black}\thmname{#1}~\thmnumber{#2}}\thmnote{\,--\,#3}}
\newtheoremstyle{myproblemstyle}{3pt}{3pt}{\normalfont}{0cm}{\rmfamily\bfseries}{}{1em}{{\color{black}\thmname{#1}~\thmnumber{#2}}\thmnote{\,--\,#3}}
\theoremstyle{mytheoremstyle}
\newmdtheoremenv[linewidth=1pt,backgroundcolor=shallowGreen,linecolor=deepGreen,leftmargin=0pt,innerleftmargin=20pt,innerrightmargin=20pt,]{theorem}{Theorem}[section]
\theoremstyle{mytheoremstyle}
\newmdtheoremenv[linewidth=1pt,backgroundcolor=shallowBlue,linecolor=deepBlue,leftmargin=0pt,innerleftmargin=20pt,innerrightmargin=20pt,]{definition}{Definition}[section]
\theoremstyle{myproblemstyle}
\newmdtheoremenv[linecolor=black,leftmargin=0pt,innerleftmargin=10pt,innerrightmargin=10pt,]{problem}{Problem}[section]
%%%%%%%%%%%%%%%%%%%%%%%%%%%%%%%%%%%%%%%%%%%%%%%%%%%%%%%%%%%%%%%%%%%%%%%%%%%%%%%

%%%%%%%%%%%%%%%%%%%%%%%%%%%%%%% Plotting Settings %%%%%%%%%%%%%%%%%%%%%%%%%%%%%
\usepgfplotslibrary{colorbrewer}
\pgfplotsset{width=8cm,compat=1.9}
%%%%%%%%%%%%%%%%%%%%%%%%%%%%%%%%%%%%%%%%%%%%%%%%%%%%%%%%%%%%%%%%%%%%%%%%%%%%%%%

%%%%%%%%%%%%%%%%%%%%%%%%%%%%%%% Title & Author %%%%%%%%%%%%%%%%%%%%%%%%%%%%%%%%
\title{Forces and Work}
\author{Patrick Chen}
\date{Nov 4, 2024}
%%%%%%%%%%%%%%%%%%%%%%%%%%%%%%%%%%%%%%%%%%%%%%%%%%%%%%%%%%%%%%%%%%%%%%%%%%%%%%%

\begin{document}
    \maketitle

    \section*{Motions of Systems}
    \begin{itemize}
        \item Internal forces is between objects in the system.
        \item External forces are form outside the system.
    \end{itemize}

    \begin{align*}
        Ma_{cm} &= \sum_j F_{net,j} \\
                &= \sum_j F_{net,j}^{ext} + \sum_j F_{net,j}^{int} \\
                &= \sum_j F_{net,j}^{ext} + 0 \\
                &= \sum_j F_{net,j}^{ext}
    \end{align*}

    Choosing a system depending on the problem can help solve the problem. When
    choosing a system, having friction on the boundary of the system will make
    analysis more difficult. Gravity will either be a change in potential energy
    or a force doing work on a system depending if it is in the system or not.

    \section*{Work}
    Work is the change in the energy of a system due to external forces. Work is
    a signed scalar quantity measured in joules (J) in the SI system.
    \begin{align*}
        W &= \mathbf{F} \cdot \mathbf{d} = Fd\cos\theta \\
        W &= \Delta E
    \end{align*}
    For non-constant forces, the work must be integrated
    \begin{align*}
        W = \int_{x_i}^{x_f} F \cdot \ dx
    \end{align*}

    Force displacement is how far the point where the force is applied moves. In
    order for a force to do work, the point where the force is being applied
    must undergo a force displacement.

    \subsection*{Momentum and Energy}

    \begin{align*}
        \Delta p = J && \Delta E = W \\
        J=0 && \Delta E = 0 \\
        J=\sum f \Delta t && W = \sum F \Delta x
    \end{align*}

    \subsection*{Example 1}
    Suppose there is a object with a horizontal for applied on a frictionless
    surface from point $x_i$ to $x_f$.
    \begin{align*}
        W &= \Delta E \\
        &= \Delta K + \Delta U + \Delta E_{th} \\
        &= \Delta K + 0 + 0 \\
        &= \Delta K \\
        F &= ma_{cm} \\
        &= m \frac{dv_{cm}}{dt} \\
        &= m \frac{dv_{cm}}{dx_{cm}} \frac{dx_{cm}}{dt} \\
        &= m v_{cm} \frac{dv_{cm}}{dx_{cm}} \\
        F\ dx_{cm}&= mv_{cm} dv_{cm} \\
        \int_{x_i}^{x_f} F \ dx_{cm} &= \int_{v_i}^{v_f} mv_{cm} \ dv_{cm} \\
        &= \frac{1}{2} mv_{cm}^2 \ \Big|_{v_i}^{v_f} \\
        &= \Delta K_{cm} \\
        F\Delta x_{cm} &= \Delta K_{cm}
    \end{align*}

    \subsection*{Example 2}
    Suppose there is a object with two horizontal forces in opposite directions
    on a frictionless surface.
    \begin{align*}
        W = \Delta K \\
        F_{net} = m a_{cm} \\
        F_{net} = F_1 - F_2 \\
        W = \int_{x_i}^{x_f} (F_1-F_2) \ dx_{cm} \\
        W_{net} = \int_{x_i}^{x_f} F_1 \ dx_{cm}
        -  \int_{x_i}^{x_f} F_2 \ dx_{cm}
    \end{align*}

    \subsection*{Example 3}
    Suppose there is a object sliding with a friction force. $F_k$ is a constant
    force. We can choose the system to be the object and the surface.
    \begin{align*}
        \Delta E &= \Delta K + \Delta E_{th} \\
        \Delta E &= 0 \\
        \Delta E_{th} &= -\Delta K \\
        &= -(K_f - K_i) \\
        &= K_i \\
        \Delta K_{cm} &= - \int_{x_{cm,i}}^{x_{cm,f}} f_x \ dx_{cm} \\
                      &= -f_k\Delta x_cm
    \end{align*}

\end{document}
