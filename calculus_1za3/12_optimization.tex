%%%%%%%%%%%%%%%%%%%%%%%%%%%%% Define Article %%%%%%%%%%%%%%%%%%%%%%%%%%%%%%%%%%
\documentclass{article}
%%%%%%%%%%%%%%%%%%%%%%%%%%%%%%%%%%%%%%%%%%%%%%%%%%%%%%%%%%%%%%%%%%%%%%%%%%%%%%%

%%%%%%%%%%%%%%%%%%%%%%%%%%%%% Using Packages %%%%%%%%%%%%%%%%%%%%%%%%%%%%%%%%%%
\usepackage{geometry}
\usepackage{graphicx}
\usepackage{amssymb}
\usepackage{amsmath}
\usepackage{amsthm}
\usepackage{empheq}
\usepackage{mdframed}
\usepackage{booktabs}
\usepackage{lipsum}
\usepackage{graphicx}
\usepackage{color}
\usepackage{psfrag}
\usepackage{pgfplots}
\usepackage{bm}
%%%%%%%%%%%%%%%%%%%%%%%%%%%%%%%%%%%%%%%%%%%%%%%%%%%%%%%%%%%%%%%%%%%%%%%%%%%%%%%

% Other Settings

%%%%%%%%%%%%%%%%%%%%%%%%%% Page Setting %%%%%%%%%%%%%%%%%%%%%%%%%%%%%%%%%%%%%%%
\geometry{a4paper}

%%%%%%%%%%%%%%%%%%%%%%%%%% Define some useful colors %%%%%%%%%%%%%%%%%%%%%%%%%%
\definecolor{ocre}{RGB}{243,102,25}
\definecolor{mygray}{RGB}{243,243,244}
\definecolor{deepGreen}{RGB}{26,111,0}
\definecolor{shallowGreen}{RGB}{235,255,255}
\definecolor{deepBlue}{RGB}{61,124,222}
\definecolor{shallowBlue}{RGB}{235,249,255}
%%%%%%%%%%%%%%%%%%%%%%%%%%%%%%%%%%%%%%%%%%%%%%%%%%%%%%%%%%%%%%%%%%%%%%%%%%%%%%%

%%%%%%%%%%%%%%%%%%%%%%%%%% Define an orangebox command %%%%%%%%%%%%%%%%%%%%%%%%
\newcommand\orangebox[1]{\fcolorbox{ocre}{mygray}{\hspace{1em}#1\hspace{1em}}}
%%%%%%%%%%%%%%%%%%%%%%%%%%%%%%%%%%%%%%%%%%%%%%%%%%%%%%%%%%%%%%%%%%%%%%%%%%%%%%%

%%%%%%%%%%%%%%%%%%%%%%%%%%%% English Environments %%%%%%%%%%%%%%%%%%%%%%%%%%%%%
\newtheoremstyle{mytheoremstyle}{3pt}{3pt}{\normalfont}{0cm}{\rmfamily\bfseries}{}{1em}{{\color{black}\thmname{#1}~\thmnumber{#2}}\thmnote{\,--\,#3}}
\newtheoremstyle{myproblemstyle}{3pt}{3pt}{\normalfont}{0cm}{\rmfamily\bfseries}{}{1em}{{\color{black}\thmname{#1}~\thmnumber{#2}}\thmnote{\,--\,#3}}
\theoremstyle{mytheoremstyle}
\newmdtheoremenv[linewidth=1pt,backgroundcolor=shallowGreen,linecolor=deepGreen,leftmargin=0pt,innerleftmargin=20pt,innerrightmargin=20pt,]{theorem}{Theorem}[section]
\theoremstyle{mytheoremstyle}
\newmdtheoremenv[linewidth=1pt,backgroundcolor=shallowBlue,linecolor=deepBlue,leftmargin=0pt,innerleftmargin=20pt,innerrightmargin=20pt,]{definition}{Definition}[section]
\theoremstyle{myproblemstyle}
\newmdtheoremenv[linecolor=black,leftmargin=0pt,innerleftmargin=10pt,innerrightmargin=10pt,]{problem}{Problem}[section]
%%%%%%%%%%%%%%%%%%%%%%%%%%%%%%%%%%%%%%%%%%%%%%%%%%%%%%%%%%%%%%%%%%%%%%%%%%%%%%%

%%%%%%%%%%%%%%%%%%%%%%%%%%%%%%% Plotting Settings %%%%%%%%%%%%%%%%%%%%%%%%%%%%%
\usepgfplotslibrary{colorbrewer}
\pgfplotsset{width=8cm,compat=1.9}
%%%%%%%%%%%%%%%%%%%%%%%%%%%%%%%%%%%%%%%%%%%%%%%%%%%%%%%%%%%%%%%%%%%%%%%%%%%%%%%

%%%%%%%%%%%%%%%%%%%%%%%%%%%%%%% Title & Author %%%%%%%%%%%%%%%%%%%%%%%%%%%%%%%%
\title{Optimization}
\author{Patrick Chen}
\date{Oct 24, 2024}
%%%%%%%%%%%%%%%%%%%%%%%%%%%%%%%%%%%%%%%%%%%%%%%%%%%%%%%%%%%%%%%%%%%%%%%%%%%%%%%

\begin{document}
    \maketitle
    When solving optimization problems, it is often beneficial to draw a picture
    representing the problem. After translating the problem into math equations
    and solving, make sure the solutions are reasonable given the context.

    \section*{Example}
    A farmer has 2400 ft of fencing and wants to fence off a rectangular field
    that borders a straight river. He needs no fence along the river. What are
    the dimensions of the field that has the largest area.

    \begin{align*}
        2x + y &= 2400 \\
        y &= 2400 - 2x \\
        A &= x\cdot y \\
        A(x) &= x(2400-2x) \\
        A(x) &= -2x^2+x2400 \\
        A'(x) &= -4x+2400 \\
        A'(x)&= 0 \\
        -4x+2400 &= 0 \\
        x &= \frac{-2400}{-4} \\
        x &= 600 \\
        2x + y &= 2400 \\
        2(600) + y &= 2400 \\
        y &= 1200
    \end{align*}

    \section*{Example 2}
    Find the area of the largest rectangle that can be inscribed in a semisircle
    of radius $r$.

    \begin{align*}
        x^2 + y^2 &= r^2 \\
        A &= 2xy \\
    \end{align*}
    \begin{align*}
        y &= \sqrt{r^2-x^2} \\
        A &= 2x\sqrt{r^2-x^2} \\
        A &= \sqrt{4x^2(r^2-x^2)} \\
        A &= \sqrt{4x^2r^2-4x^4}
    \end{align*}

    Since the maximum of $\sqrt{f(x)}$ is equal to $\sqrt{\text{max of }f(x)}$,
    we can find the max of $4x^2r^2-4x^4$, then take a square root. This works
    in general for any increasing function.
    \begin{align*}
        f(x) &=4x^2r^2-4x^4 \\
        f'(x) &= 8xr^2 - 16x^3 \\
        f'(x) &= 8x(r^2 - 2x^2) \\
        f'(x) &= 8x(r-x\sqrt{2})(r+x\sqrt{2}) \\
        f'(x) &= -8x(x\sqrt{2}-r)(x\sqrt{2}+r) \\
        x &= 0, \frac{r}{\sqrt{2}}, -\frac{r}{\sqrt{2}}
    \end{align*}
    since $x$ must be positive, $x=\frac{r}{\sqrt{2}}$.

    \section*{Example 3}
    Find the point on the parabola $y^2=2x$ closest to $(1,4)$.
    \begin{align*}
        y^2&=2x \\
        r  &= \sqrt{(x-1)^2+(y-4)^2}
    \end{align*}
    \begin{align*}
        x &= \frac{y^2}{2} \\
        r &= \sqrt{(\frac{1}{2}y^2-1)^2+(y-4)^2} \\
        r &= \sqrt{\frac{1}{4} y^4 -y^2+1 + y^2-8y+16} \\
        r &= \sqrt{\frac{1}{4} y^4 -8y + 17} \\
        f(y) &= \frac{1}{4} y^4 -8y + 17 \\
        f'(y) &= y^3 -8 \\
        0 &= y^3 -8 \\
        8 &= y^3 \\
        2 &= y
    \end{align*}
\end{document}
