%%%%%%%%%%%%%%%%%%%%%%%%%%%%% Define Article %%%%%%%%%%%%%%%%%%%%%%%%%%%%%%%%%%
\documentclass{article}
%%%%%%%%%%%%%%%%%%%%%%%%%%%%%%%%%%%%%%%%%%%%%%%%%%%%%%%%%%%%%%%%%%%%%%%%%%%%%%%

%%%%%%%%%%%%%%%%%%%%%%%%%%%%% Using Packages %%%%%%%%%%%%%%%%%%%%%%%%%%%%%%%%%%
\usepackage{geometry}
\usepackage{graphicx}
\usepackage{amssymb}
\usepackage{amsmath}
\usepackage{amsthm}
\usepackage{empheq}
\usepackage{mdframed}
\usepackage{booktabs}
\usepackage{lipsum}
\usepackage{graphicx}
\usepackage{color}
\usepackage{psfrag}
\usepackage{pgfplots}
\usepackage{bm}
%%%%%%%%%%%%%%%%%%%%%%%%%%%%%%%%%%%%%%%%%%%%%%%%%%%%%%%%%%%%%%%%%%%%%%%%%%%%%%%

% Other Settings

%%%%%%%%%%%%%%%%%%%%%%%%%% Page Setting %%%%%%%%%%%%%%%%%%%%%%%%%%%%%%%%%%%%%%%
\geometry{a4paper}

%%%%%%%%%%%%%%%%%%%%%%%%%% Define some useful colors %%%%%%%%%%%%%%%%%%%%%%%%%%
\definecolor{ocre}{RGB}{243,102,25}
\definecolor{mygray}{RGB}{243,243,244}
\definecolor{deepGreen}{RGB}{26,111,0}
\definecolor{shallowGreen}{RGB}{235,255,255}
\definecolor{deepBlue}{RGB}{61,124,222}
\definecolor{shallowBlue}{RGB}{235,249,255}
%%%%%%%%%%%%%%%%%%%%%%%%%%%%%%%%%%%%%%%%%%%%%%%%%%%%%%%%%%%%%%%%%%%%%%%%%%%%%%%

%%%%%%%%%%%%%%%%%%%%%%%%%% Define an orangebox command %%%%%%%%%%%%%%%%%%%%%%%%
\newcommand\orangebox[1]{\fcolorbox{ocre}{mygray}{\hspace{1em}#1\hspace{1em}}}
%%%%%%%%%%%%%%%%%%%%%%%%%%%%%%%%%%%%%%%%%%%%%%%%%%%%%%%%%%%%%%%%%%%%%%%%%%%%%%%

%%%%%%%%%%%%%%%%%%%%%%%%%%%% English Environments %%%%%%%%%%%%%%%%%%%%%%%%%%%%%
\newtheoremstyle{mytheoremstyle}{3pt}{3pt}{\normalfont}{0cm}{\rmfamily\bfseries}{}{1em}{{\color{black}\thmname{#1}~\thmnumber{#2}}\thmnote{\,--\,#3}}
\newtheoremstyle{myproblemstyle}{3pt}{3pt}{\normalfont}{0cm}{\rmfamily\bfseries}{}{1em}{{\color{black}\thmname{#1}~\thmnumber{#2}}\thmnote{\,--\,#3}}
\theoremstyle{mytheoremstyle}
\newmdtheoremenv[linewidth=1pt,backgroundcolor=shallowGreen,linecolor=deepGreen,leftmargin=0pt,innerleftmargin=20pt,innerrightmargin=20pt,]{theorem}{Theorem}[section]
\theoremstyle{mytheoremstyle}
\newmdtheoremenv[linewidth=1pt,backgroundcolor=shallowBlue,linecolor=deepBlue,leftmargin=0pt,innerleftmargin=20pt,innerrightmargin=20pt,]{definition}{Definition}[section]
\theoremstyle{myproblemstyle}
\newmdtheoremenv[linecolor=black,leftmargin=0pt,innerleftmargin=10pt,innerrightmargin=10pt,]{problem}{Problem}[section]
%%%%%%%%%%%%%%%%%%%%%%%%%%%%%%%%%%%%%%%%%%%%%%%%%%%%%%%%%%%%%%%%%%%%%%%%%%%%%%%

%%%%%%%%%%%%%%%%%%%%%%%%%%%%%%% Plotting Settings %%%%%%%%%%%%%%%%%%%%%%%%%%%%%
\usepgfplotslibrary{colorbrewer}
\pgfplotsset{width=8cm,compat=1.9}
%%%%%%%%%%%%%%%%%%%%%%%%%%%%%%%%%%%%%%%%%%%%%%%%%%%%%%%%%%%%%%%%%%%%%%%%%%%%%%%

%%%%%%%%%%%%%%%%%%%%%%%%%%%%%%% Title & Author %%%%%%%%%%%%%%%%%%%%%%%%%%%%%%%%
\title{Evaluating Integrals}
\author{Patrick Chen}
\date{Nov 7, 2024}
%%%%%%%%%%%%%%%%%%%%%%%%%%%%%%%%%%%%%%%%%%%%%%%%%%%%%%%%%%%%%%%%%%%%%%%%%%%%%%%

\begin{document}
    \maketitle
    The rules of derivatives can be inverted into rules of integrals.

    \section*{Substitution Rule}
    The substitution rule (also called u-substitution) of the integral is the
    reverse of the chain rule for derivative. The function $f$ must be
    continuous on the range of $g(x)=u$.
    \begin{align*}
        (F(u(x)))' &= f(u(x))u'(x) \\
        F(u(x)) &= \int f(u(x))u'(x) \ dx
    \end{align*}

    When using substitution for definite integrals, the bounds needs to be
    changed to account for the change in integration variable. The bounds can be
    found by substituting the old bounds in for $x$ and solving for $u$.
    \begin{align*}
        \int_{a}^{b} f(u(x)) u'(x) \ dx
        = \int_{u(a)}^{u(b)} f(u) \ du
    \end{align*}

    \subsection*{Example 1}
    Evaluate $\int 2x\sqrt{1+x^2} \ dx$
    \begin{align*}
        u &= 1+x^2 \\
        \frac{du}{dx} &= 2x \\
        du &= 2x\ dx \\
        \int 2x\sqrt{1+x^2} \ dx &= \int \sqrt{u}\ 2x\ dx \\
        &= \int \sqrt{u} \ du \\
        &= \int u^{\frac{1}{2}} \ du \\
        &= \frac{2}{3} u^{\frac{3}{2}} + c \\
        &= \frac{2}{3} (1+x^2)^{\frac{3}{2}} + c
    \end{align*}

    \subsection*{Example 2}
    Evaluate $\int x^3\cos(x^4+2) \ dx$
    \begin{align*}
        u  &= x^4 + 2 \\
        du &= 4x^3 dx \\
        \int x^3\cos(x^4+2) \ dx
        &= \int \frac{4x^3}{4} \cos(u) \ dx \\
        &= \frac{1}{4} \int \cos(u) \ du \\
        &= \frac{1}{4} \sin(u) + c \\
        &= \frac{1}{4} \sin(x^4 + 2) + c
    \end{align*}

    \subsection*{Example 3}
    Evaluate $\int \tan x \ dx$
    \begin{align*}
        u &= \cos x \\
        du &= -\sin x dx \\
        \int \tan x \ dx
        &= \int \frac{\sin x}{\cos x} \ dx \\
        &= \int -\frac{1}{u} \ du \\
        &= -\ln |u| \\
        &= -\ln |\cos x|
    \end{align*}

    \subsection*{Example 4}
    Evaluate $\int_{0}^{1} \frac{1}{x+4} \ dx$.
    \begin{align*}
        u &= x+4 \\
        du &= dx \\
        \int_{0}^{1} \frac{1}{x+4} \ dx
        &= \int_{(0)+4}^{(1)+4} \frac{1}{u} \ du \\
        &= \int_{4}^{5} \frac{1}{u} \ du \\
        &= \ln|u| \ \Big|_{4}^{5} \\
        &= \ln|5| - \ln|4| \\
        &= \ln(\frac{5}{4})
    \end{align*}

\end{document}
