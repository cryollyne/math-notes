%%%%%%%%%%%%%%%%%%%%%%%%%%%%% Define Article %%%%%%%%%%%%%%%%%%%%%%%%%%%%%%%%%%
\documentclass{article}
%%%%%%%%%%%%%%%%%%%%%%%%%%%%%%%%%%%%%%%%%%%%%%%%%%%%%%%%%%%%%%%%%%%%%%%%%%%%%%%

%%%%%%%%%%%%%%%%%%%%%%%%%%%%% Using Packages %%%%%%%%%%%%%%%%%%%%%%%%%%%%%%%%%%
\usepackage{geometry}
\usepackage{graphicx}
\usepackage{amssymb}
\usepackage{amsmath}
\usepackage{amsthm}
\usepackage{empheq}
\usepackage{mdframed}
\usepackage{booktabs}
\usepackage{lipsum}
\usepackage{graphicx}
\usepackage{color}
\usepackage{psfrag}
\usepackage{pgfplots}
\usepackage{bm}
%%%%%%%%%%%%%%%%%%%%%%%%%%%%%%%%%%%%%%%%%%%%%%%%%%%%%%%%%%%%%%%%%%%%%%%%%%%%%%%

% Other Settings

%%%%%%%%%%%%%%%%%%%%%%%%%% Page Setting %%%%%%%%%%%%%%%%%%%%%%%%%%%%%%%%%%%%%%%
\geometry{a4paper}

%%%%%%%%%%%%%%%%%%%%%%%%%% Define some useful colors %%%%%%%%%%%%%%%%%%%%%%%%%%
\definecolor{ocre}{RGB}{243,102,25}
\definecolor{mygray}{RGB}{243,243,244}
\definecolor{deepGreen}{RGB}{26,111,0}
\definecolor{shallowGreen}{RGB}{235,255,255}
\definecolor{deepBlue}{RGB}{61,124,222}
\definecolor{shallowBlue}{RGB}{235,249,255}
%%%%%%%%%%%%%%%%%%%%%%%%%%%%%%%%%%%%%%%%%%%%%%%%%%%%%%%%%%%%%%%%%%%%%%%%%%%%%%%

%%%%%%%%%%%%%%%%%%%%%%%%%% Define an orangebox command %%%%%%%%%%%%%%%%%%%%%%%%
\newcommand\orangebox[1]{\fcolorbox{ocre}{mygray}{\hspace{1em}#1\hspace{1em}}}
%%%%%%%%%%%%%%%%%%%%%%%%%%%%%%%%%%%%%%%%%%%%%%%%%%%%%%%%%%%%%%%%%%%%%%%%%%%%%%%

%%%%%%%%%%%%%%%%%%%%%%%%%%%% English Environments %%%%%%%%%%%%%%%%%%%%%%%%%%%%%
\newtheoremstyle{mytheoremstyle}{3pt}{3pt}{\normalfont}{0cm}{\rmfamily\bfseries}{}{1em}{{\color{black}\thmname{#1}~\thmnumber{#2}}\thmnote{\,--\,#3}}
\newtheoremstyle{myproblemstyle}{3pt}{3pt}{\normalfont}{0cm}{\rmfamily\bfseries}{}{1em}{{\color{black}\thmname{#1}~\thmnumber{#2}}\thmnote{\,--\,#3}}
\theoremstyle{mytheoremstyle}
\newmdtheoremenv[linewidth=1pt,backgroundcolor=shallowGreen,linecolor=deepGreen,leftmargin=0pt,innerleftmargin=20pt,innerrightmargin=20pt,]{theorem}{Theorem}[section]
\theoremstyle{mytheoremstyle}
\newmdtheoremenv[linewidth=1pt,backgroundcolor=shallowBlue,linecolor=deepBlue,leftmargin=0pt,innerleftmargin=20pt,innerrightmargin=20pt,]{definition}{Definition}[section]
\theoremstyle{myproblemstyle}
\newmdtheoremenv[linecolor=black,leftmargin=0pt,innerleftmargin=10pt,innerrightmargin=10pt,]{problem}{Problem}[section]
%%%%%%%%%%%%%%%%%%%%%%%%%%%%%%%%%%%%%%%%%%%%%%%%%%%%%%%%%%%%%%%%%%%%%%%%%%%%%%%

%%%%%%%%%%%%%%%%%%%%%%%%%%%%%%% Plotting Settings %%%%%%%%%%%%%%%%%%%%%%%%%%%%%
\usepgfplotslibrary{colorbrewer}
\pgfplotsset{width=8cm,compat=1.9}
%%%%%%%%%%%%%%%%%%%%%%%%%%%%%%%%%%%%%%%%%%%%%%%%%%%%%%%%%%%%%%%%%%%%%%%%%%%%%%%

%%%%%%%%%%%%%%%%%%%%%%%%%%%%%%% Title & Author %%%%%%%%%%%%%%%%%%%%%%%%%%%%%%%%
\title{Trigonometry Review}
\author{Patrick Chen}
\date{Sept 4, 2024}
%%%%%%%%%%%%%%%%%%%%%%%%%%%%%%%%%%%%%%%%%%%%%%%%%%%%%%%%%%%%%%%%%%%%%%%%%%%%%%%

\begin{document}
    \maketitle
    \section*{Angles}
    Positive angles is counter clockwise, negative angles are clockwise. In this
    course, we will use radians. Radians is a measure of how many times a radius
    fit in the arc corresponding with the angle.
    \begin{align*}
        \pi\ radian &= 180^\circ \\
        \frac{\pi D}{180^\circ} &= rad
    \end{align*}

    Note that $-\frac{5\pi}{4}$ is not the same as $\frac{3\pi}{4}$ even though
    it ends in the same place because one is clockwise and the other is counter
    clockwise

    \section*{Acute angles}
    Trig functions can be defined with right triangles
    \begin{equation*}
        \begin{matrix}
            sin \theta = \frac{opp}{hyp} & cos \theta = \frac{adj}{hyp} & tan \theta = \frac{opp}{adj} \\
            csc\theta=\frac{1 }{sin\theta} & sec\theta=\frac{1 }{soc\theta} & cot\theta = \frac{1 }{tan\theta}
        \end{matrix}
    \end{equation*}

    Special triangles:
    ($1, 1, \frac{\sqrt{2}}{2}$), ($1, 2, \sqrt{3}$)

    \section*{Obtuse and Negative angles}
    A point with distance $r$ away from the origin and angle $\theta$ with have
    an x coordinate of $rcos\theta$ and a y coordinate of $rsin\theta$
    \begin{equation*}
        \begin{matrix}
            sin \theta = \frac{y}{r} & cos \theta = \frac{x}{r} & tan \theta = \frac{y}{x} \\
            csc\theta=\frac{1}{sin\theta} & sec\theta=\frac{1}{soc\theta} & cot\theta = \frac{1}{tan\theta}
        \end{matrix}
    \end{equation*}

    \section*{Trig Identities}
    \begin{align*}
        sin^2\theta + cos^2\theta = 1 \\
        1 + tan^2\theta = sec^2\theta \\
        1 + cot^2\theta = csc^2\theta
    \end{align*}

%     \begin{align*}
%         & 1 + tan2\theta \\
%         =& \frac{cos^2}{cos^2} + \frac{sin^2\theta}{cos^2\theta} \\
%         =& \frac{cos^2\theta + sin^2\theta}{cos^2\theta} \\
%         =& \frac{1}{cos^2\theta} \\
%         =&sec^2\theta
%     \end{align*}

    Sine is the y axis coordinate so a negating the angle will negate the sine
    of the angle. Cosine is the x axis coordinate which doesn't change when the
    angle is negated
    \begin{align*}
        sin(-\theta) = -sin(\theta) \\
        cos(-\theta) = cos(\theta)
    \end{align*}

    Sine and Cosine is a quarter rotation phase shifted
    \begin{align*}
        cos\theta = sin(\theta + \frac{\pi}{2}) \\
        sin\theta = cos(\theta - \frac{\pi}{2})
    \end{align*}

    \begin{align*}
        \begin{matrix}
            \theta & 0 & \frac{\pi}{2} & \pi & \frac{3\pi}{2} & 2\pi \\
            sin\theta & 0 & 1 & 0 & -1 & 0 \\
            cos\theta & 1 & 0 & -1 & 0 & 1 \\
            tan\theta & 0 & ud & 0 & ud & 0
        \end{matrix}
    \end{align*}
\end{document}
