%%%%%%%%%%%%%%%%%%%%%%%%%%%%% Define Article %%%%%%%%%%%%%%%%%%%%%%%%%%%%%%%%%%
\documentclass{article}
%%%%%%%%%%%%%%%%%%%%%%%%%%%%%%%%%%%%%%%%%%%%%%%%%%%%%%%%%%%%%%%%%%%%%%%%%%%%%%%

%%%%%%%%%%%%%%%%%%%%%%%%%%%%% Using Packages %%%%%%%%%%%%%%%%%%%%%%%%%%%%%%%%%%
\usepackage{geometry}
\usepackage{graphicx}
\usepackage{amssymb}
\usepackage{amsmath}
\usepackage{amsthm}
\usepackage{empheq}
\usepackage{mdframed}
\usepackage{booktabs}
\usepackage{lipsum}
\usepackage{graphicx}
\usepackage{color}
\usepackage{psfrag}
\usepackage{pgfplots}
\usepackage{bm}
%%%%%%%%%%%%%%%%%%%%%%%%%%%%%%%%%%%%%%%%%%%%%%%%%%%%%%%%%%%%%%%%%%%%%%%%%%%%%%%

% Other Settings

%%%%%%%%%%%%%%%%%%%%%%%%%% Page Setting %%%%%%%%%%%%%%%%%%%%%%%%%%%%%%%%%%%%%%%
\geometry{a4paper}

%%%%%%%%%%%%%%%%%%%%%%%%%% Define some useful colors %%%%%%%%%%%%%%%%%%%%%%%%%%
\definecolor{ocre}{RGB}{243,102,25}
\definecolor{mygray}{RGB}{243,243,244}
\definecolor{deepGreen}{RGB}{26,111,0}
\definecolor{shallowGreen}{RGB}{235,255,255}
\definecolor{deepBlue}{RGB}{61,124,222}
\definecolor{shallowBlue}{RGB}{235,249,255}
%%%%%%%%%%%%%%%%%%%%%%%%%%%%%%%%%%%%%%%%%%%%%%%%%%%%%%%%%%%%%%%%%%%%%%%%%%%%%%%

%%%%%%%%%%%%%%%%%%%%%%%%%% Define an orangebox command %%%%%%%%%%%%%%%%%%%%%%%%
\newcommand\orangebox[1]{\fcolorbox{ocre}{mygray}{\hspace{1em}#1\hspace{1em}}}
%%%%%%%%%%%%%%%%%%%%%%%%%%%%%%%%%%%%%%%%%%%%%%%%%%%%%%%%%%%%%%%%%%%%%%%%%%%%%%%

%%%%%%%%%%%%%%%%%%%%%%%%%%%% English Environments %%%%%%%%%%%%%%%%%%%%%%%%%%%%%
\newtheoremstyle{mytheoremstyle}{3pt}{3pt}{\normalfont}{0cm}{\rmfamily\bfseries}{}{1em}{{\color{black}\thmname{#1}~\thmnumber{#2}}\thmnote{\,--\,#3}}
\newtheoremstyle{myproblemstyle}{3pt}{3pt}{\normalfont}{0cm}{\rmfamily\bfseries}{}{1em}{{\color{black}\thmname{#1}~\thmnumber{#2}}\thmnote{\,--\,#3}}
\theoremstyle{mytheoremstyle}
\newmdtheoremenv[linewidth=1pt,backgroundcolor=shallowGreen,linecolor=deepGreen,leftmargin=0pt,innerleftmargin=20pt,innerrightmargin=20pt,]{theorem}{Theorem}[section]
\theoremstyle{mytheoremstyle}
\newmdtheoremenv[linewidth=1pt,backgroundcolor=shallowBlue,linecolor=deepBlue,leftmargin=0pt,innerleftmargin=20pt,innerrightmargin=20pt,]{definition}{Definition}[section]
\theoremstyle{myproblemstyle}
\newmdtheoremenv[linecolor=black,leftmargin=0pt,innerleftmargin=10pt,innerrightmargin=10pt,]{problem}{Problem}[section]
%%%%%%%%%%%%%%%%%%%%%%%%%%%%%%%%%%%%%%%%%%%%%%%%%%%%%%%%%%%%%%%%%%%%%%%%%%%%%%%

%%%%%%%%%%%%%%%%%%%%%%%%%%%%%%% Plotting Settings %%%%%%%%%%%%%%%%%%%%%%%%%%%%%
\usepgfplotslibrary{colorbrewer}
\pgfplotsset{width=8cm,compat=1.9}
%%%%%%%%%%%%%%%%%%%%%%%%%%%%%%%%%%%%%%%%%%%%%%%%%%%%%%%%%%%%%%%%%%%%%%%%%%%%%%%

%%%%%%%%%%%%%%%%%%%%%%%%%%%%%%% Title & Author %%%%%%%%%%%%%%%%%%%%%%%%%%%%%%%%
\title{Indeterminate Forms}
\author{Patrick Chen}
\date{Oct 10, 2024}
%%%%%%%%%%%%%%%%%%%%%%%%%%%%%%%%%%%%%%%%%%%%%%%%%%%%%%%%%%%%%%%%%%%%%%%%%%%%%%%

\newcommand\lH{\stackrel{H}{=}}

\begin{document}
    \maketitle
    Indeterminate forms are forms where substituting the limit variable will not
    give an answer.
    
    \section*{Indeterminate Fractions}
    \begin{align*}
        \frac{0}{0} &&
        \frac{\infty}{\infty}
    \end{align*}

    L'Hospital's rule is a rule that can be used if the numerator and
    denominator of a fraction are both zero or both infinite.

    \begin{align*}
        \lim_{x\to a} f(x)=0 &\text{ and } \lim_{x\to a} g(x)=0 \\
        &\text{  or} \\
        \lim_{x\to a} f(x)=\infty &\text{ and } \lim_{x\to a} g(x)=\infty \\
                                  &\text{then} \\
        \lim_{x\to a} \frac{f(x)}{g(x)} &= \lim_{x\to a} \frac{f'(x)}{g'(x)}
    \end{align*}

    \subsection*{Example 1}
    \begin{align*}
        &\lim_{x\to \infty} \frac{e^x}{x^2} = \frac{\infty}{\infty} \\
        \lH&\lim_{x\to \infty} \frac{e^x}{2x} \\
        \lH&\lim_{x\to \infty} \frac{e^x}{2} = \frac{\infty}{2} = \infty
    \end{align*}

    \section*{Indeterminate products}
    \begin{align*}
        & 0\cdot \infty \\
    \end{align*}
    When a indeterminate product is found, it is often favorable to rewrite it
    as a fraction.
    \begin{align*}
        \frac{0}{(\frac{1}{\infty})}=\frac{0}{0}
        && \frac{\infty}{(\frac{1}{0})} = \frac{\infty}{\infty}
    \end{align*}

    \subsection*{Example 2}
    \begin{align*}
        & \lim_{x\to 0^+} xe^{\frac{1}{x}} = 0\cdot\infty \\
        =& \lim_{x\to 0^+} \frac{e^{\frac{1}{x}}}{\frac{1}{x}} \\
        \lH& \lim_{x\to 0^+} \frac{-\frac{1}{x^2} e^{\frac{1}{x}}}{-\frac{1}{x^2}} \\
        =& \lim_{x\to 0^+} e^{\frac{1}{x}} \\
        =& e^{\frac{1}{0}} \\
        =& e^{\infty} \\
        =& \infty
    \end{align*}

    \subsection*{Example 3}
    \begin{align*}
        & \lim_{x\to 0^+} x\ln x = 0\cdot-\infty \\
        =& \lim_{x\to 0^+} \frac{\ln x}{\frac{1}{x}}\\
        \lH& \lim_{x\to 0^+} \frac{\frac{1}{x}}{-\frac{1}{x^2}} \\
        =& \lim_{x\to 0^+} (-x) \\
        =& 0
    \end{align*}

    \section*{Indeterminate Powers}
    \begin{align*}
        0^0 \qquad
        \infty^0 \qquad
        1^\infty
    \end{align*}
    Indeterminate powers can be converted into products by applying a logarithm.
    \begin{align*}
        \lim_{x\to n} a^b = e^{\big(\lim\limits_{x\to n} b\cdot \ln a\big)}
    \end{align*}

    \subsection*{Example 4}
    \begin{align*}
        L =& \lim_{x\to 0^+} (1+\sin 4x)^{\cot x} = 1^\infty \\
        \ln L =& \lim_{x\to 0^+} \cot x \ln(1+\sin 4x) = \infty \cdot 0 \\
        \ln L =& \lim_{x\to 0^+} \frac{ln(1+\sin 4x)}{\tan x} \\
        \ln L \lH& \lim_{x\to 0^+} \frac{(\frac{4\cos 4x}{1+\sin 4x})}{1+\tan^2 x} \\
        \ln L =& \frac{(\frac{4\cos 0}{1+\sin 0})}{1+\tan^2 0} \\
        \ln L =& \frac{4}{1} \\
        \ln L =& 4 \\
        L   =& e^4
    \end{align*}
\end{document}
