%%%%%%%%%%%%%%%%%%%%%%%%%%%%% Define Article %%%%%%%%%%%%%%%%%%%%%%%%%%%%%%%%%%
\documentclass{article}
%%%%%%%%%%%%%%%%%%%%%%%%%%%%%%%%%%%%%%%%%%%%%%%%%%%%%%%%%%%%%%%%%%%%%%%%%%%%%%%

%%%%%%%%%%%%%%%%%%%%%%%%%%%%% Using Packages %%%%%%%%%%%%%%%%%%%%%%%%%%%%%%%%%%
\usepackage{geometry}
\usepackage{graphicx}
\usepackage{amssymb}
\usepackage{amsmath}
\usepackage{amsthm}
\usepackage{empheq}
\usepackage{mdframed}
\usepackage{booktabs}
\usepackage{lipsum}
\usepackage{graphicx}
\usepackage{color}
\usepackage{psfrag}
\usepackage{pgfplots}
\usepackage{bm}
%%%%%%%%%%%%%%%%%%%%%%%%%%%%%%%%%%%%%%%%%%%%%%%%%%%%%%%%%%%%%%%%%%%%%%%%%%%%%%%

% Other Settings

%%%%%%%%%%%%%%%%%%%%%%%%%% Page Setting %%%%%%%%%%%%%%%%%%%%%%%%%%%%%%%%%%%%%%%
\geometry{a4paper}

%%%%%%%%%%%%%%%%%%%%%%%%%% Define some useful colors %%%%%%%%%%%%%%%%%%%%%%%%%%
\definecolor{ocre}{RGB}{243,102,25}
\definecolor{mygray}{RGB}{243,243,244}
\definecolor{deepGreen}{RGB}{26,111,0}
\definecolor{shallowGreen}{RGB}{235,255,255}
\definecolor{deepBlue}{RGB}{61,124,222}
\definecolor{shallowBlue}{RGB}{235,249,255}
%%%%%%%%%%%%%%%%%%%%%%%%%%%%%%%%%%%%%%%%%%%%%%%%%%%%%%%%%%%%%%%%%%%%%%%%%%%%%%%

%%%%%%%%%%%%%%%%%%%%%%%%%% Define an orangebox command %%%%%%%%%%%%%%%%%%%%%%%%
\newcommand\orangebox[1]{\fcolorbox{ocre}{mygray}{\hspace{1em}#1\hspace{1em}}}
%%%%%%%%%%%%%%%%%%%%%%%%%%%%%%%%%%%%%%%%%%%%%%%%%%%%%%%%%%%%%%%%%%%%%%%%%%%%%%%

%%%%%%%%%%%%%%%%%%%%%%%%%%%% English Environments %%%%%%%%%%%%%%%%%%%%%%%%%%%%%
\newtheoremstyle{mytheoremstyle}{3pt}{3pt}{\normalfont}{0cm}{\rmfamily\bfseries}{}{1em}{{\color{black}\thmname{#1}~\thmnumber{#2}}\thmnote{\,--\,#3}}
\newtheoremstyle{myproblemstyle}{3pt}{3pt}{\normalfont}{0cm}{\rmfamily\bfseries}{}{1em}{{\color{black}\thmname{#1}~\thmnumber{#2}}\thmnote{\,--\,#3}}
\theoremstyle{mytheoremstyle}
\newmdtheoremenv[linewidth=1pt,backgroundcolor=shallowGreen,linecolor=deepGreen,leftmargin=0pt,innerleftmargin=20pt,innerrightmargin=20pt,]{theorem}{Theorem}[section]
\theoremstyle{mytheoremstyle}
\newmdtheoremenv[linewidth=1pt,backgroundcolor=shallowBlue,linecolor=deepBlue,leftmargin=0pt,innerleftmargin=20pt,innerrightmargin=20pt,]{definition}{Definition}[section]
\theoremstyle{myproblemstyle}
\newmdtheoremenv[linecolor=black,leftmargin=0pt,innerleftmargin=10pt,innerrightmargin=10pt,]{problem}{Problem}[section]
%%%%%%%%%%%%%%%%%%%%%%%%%%%%%%%%%%%%%%%%%%%%%%%%%%%%%%%%%%%%%%%%%%%%%%%%%%%%%%%

%%%%%%%%%%%%%%%%%%%%%%%%%%%%%%% Plotting Settings %%%%%%%%%%%%%%%%%%%%%%%%%%%%%
\usepgfplotslibrary{colorbrewer}
\pgfplotsset{width=8cm,compat=1.9}
%%%%%%%%%%%%%%%%%%%%%%%%%%%%%%%%%%%%%%%%%%%%%%%%%%%%%%%%%%%%%%%%%%%%%%%%%%%%%%%

%%%%%%%%%%%%%%%%%%%%%%%%%%%%%%% Title & Author %%%%%%%%%%%%%%%%%%%%%%%%%%%%%%%%
\title{Partial Fraction}
\author{Patrick Chen}
\date{ Nov 27, 2024}
%%%%%%%%%%%%%%%%%%%%%%%%%%%%%%%%%%%%%%%%%%%%%%%%%%%%%%%%%%%%%%%%%%%%%%%%%%%%%%%

\begin{document}
    \maketitle
    \begin{align*}
        \int \frac{P(x)}{Q(x)} \ dx
    \end{align*}
    If $P(x)$ and $Q(x)$ are polynomials.

    \begin{itemize}
        \item If $deg(P)\ge deg(Q)$, then use polynomial long division.
        \item If the degree of the numerator is less than the degree of the
            denominator and $Q(x)$ is factorizable into non-repeating linear
            terms, then:
            \begin{align*}
                Q(x) = (a_1x+b_1)(a_2x+b_2)\dots(a_nxb_n) \\
                \frac{P(x)}{Q(x)} = \frac{A_1}{a_1x+b_1} + \frac{A_2}{a_2x+b_2}+\dots+\frac{A_n}{a_nxb_n} \\
            \end{align*}
        \item If the degree of the numerator is less than the degree of the
            denominator and the factorization of $Q(x)$ contains repeated roots,
            then there must be fractions for each of the degrees for that root
            \begin{align*}
                P(x) &= 4 \\
                Q(x) &= (x-3)^3(x+2) \\
                \frac{4}{(x-3)^3(x+2)}
                &= \frac{A_1}{(x-3)} + \frac{A_2}{(x-3)^2} + \frac{A_3}{(x-3)^3} + \frac{A_4}{x+2}
            \end{align*}
        \item If the denominator is a irreducible polynomial (polynomials with
            no real roots), instead of putting constants in the numerator, put a
            polynomial with one fewer degree in the numerator.

            \begin{align*}
                \frac{1}{(x-1)(x^2+x+2)} = \frac{A_1}{x-1} + \frac{A_2x+B_2}{x^2+x+2}
            \end{align*}
    \end{itemize}

    Since these equations are true for all values $x$, certain values of $x$ can
    be strategically substituted to cancel the other factors and make solving
    easier. Every polynomial can be factored into linear and quadratic terms in
    the reals.
    \begin{align*}
        x^4 + 1 &= (x^2 + ax + 1) (x^2 + bx + 1) \\
        a &= \sqrt{2} \\
        b &= -\sqrt{2}
    \end{align*}

    \subsection*{Example 1}
    \begin{align*}
        \int \frac{x^3+x}{x-1} \ dx
    \end{align*}
    \begin{align*}
        \arraycolsep=1pt
        \renewcommand\arraystretch{1.2}
        \begin{array}{*1r @{\hskip\arraycolsep}c@{\hskip\arraycolsep} *{11}r}
                  &   & x^2 & -x   & +2 &   \\
              \cline{3-6}
            x  -x & | & x^3 &      & +x &   \\
                  &   & x^3 & -x^2 &    &   \\
                  \cline{3-5}
                  &   &     & -x^2 & +x &   \\
                  &   &     &  x^2 & -x &   \\
                  \cline{4-5}
                  &   &     &     & 2x &    \\
                  &   &     &     & 2x & -2 \\
                  \cline{5-6}
                  &   &     &     &    & 2  \\
        \end{array}
    \end{align*}
    \begin{align*}
        \frac{x^3+x}{x-1} &= (x^2-x+2)+\frac{2}{x-1} \\
        \int \frac{x^3+x}{x-1} \ dx
        &= \int (x^2-x+2)+\frac{2}{x-1} \ dx \\
        &= \frac{x^3}{3} + \frac{x^2}{2} + 2x + 2\ln|x-1| + c
    \end{align*}

    \subsection*{Example 2}
    \begin{align*}
        \int \frac{x^2+2x-1}{2x^3+3x^2-2x} \ dx
    \end{align*}
    \begin{align*}
        2x^3+3x^2-2x
        &= x(2x^2+3x-2) \\
        &= x(2x^2 +4x-1x-2) \\
        &= x(x+2)(2x-1) \\
        \frac{x^2+2x-1}{2x^3+3x^2-2x}
        &= \frac{A_1}{x} + \frac{A_2}{2x+2} + \frac{A_3}{x+2} \\
        &= \frac{A_1(2x+2)(x+2) + A_2x(x+2) + A_3x(2x+2)}{2x^3+3x^2-2x} \\
        x^2+2x-1 &= A_1(2x+2)(x+2) + A_2x(x+2) + A_3x(2x+2) \\
        x^2+2x-1
        &=
        (2A_1+A_2+2A_3)x^2 + (3A_1+2A_2 - A_3)x - 2A_1
    \end{align*}
    \begin{align*}
        x^2 &= (2A_1+A_2+2A_3)x^2 \\
        2x &= (3A_1+2A_2 - A_3)x \\
        -1 &= - 2A_1 \\
        A_1 &= 0/2 \\
        A_2 &= a \\
        A_3 &= -1/10
    \end{align*}

    \subsection*{Example 3}
    \begin{align*}
        \int \frac{x^4-2x^2+4x+1}{x^3-x^2-x+1} \ dx
    \end{align*}

    \begin{align*}
        \arraycolsep=1pt
        \renewcommand\arraystretch{1.2}
        \begin{array}{*1r @{\hskip\arraycolsep}c@{\hskip\arraycolsep} *{11}r}
                        &   & x   & +1   &        &     & \\
            \cline{3-6}
            x^3-x^2-x+1 & | & x^4 &      & -2x^2  & +4x & +1 \\
                        &   & x^4 & -x^3 & -x^2   & +x  & \\
                        \cline{3-6}
                        &   &     & x^3  & -x^2   & +3x & +1  \\
                        &   &     & x^3  & -x^2   & -x  & + 1 \\
                        \cline{4-7}
                        &   &     &      &        & 2x  &
        \end{array}
    \end{align*}

    \begin{align*}
        \int x+1+ \frac{4x}{x^3-x^2-x+1} \ dx
    \end{align*}

    \begin{align*}
        \arraycolsep=1pt
        \renewcommand\arraystretch{1.2}
        \begin{array}{*1r @{\hskip\arraycolsep}c@{\hskip\arraycolsep} *{11}r}
                  &   & x^2   &       & -1  & \\
            \cline{3-6}
            (x-1) & | & x^3   & -x^2  & -x  & +1 \\
                  &   & x^3   & -x^2 \\
            \cline{3-6}
                  &   &       &       & -x  & +1 \\
                  &   &       &       & -x  & +1 \\
            \cline{5-6}
                  &   &       &       &     & 0
        \end{array}
    \end{align*}
    \begin{align*}
        \int x+1+ \frac{4x}{(x-1)(x^2-1)} \ dx
    \end{align*}

    \begin{align*}
        \frac{4x}{(x-1)(x^2-1)}
        &= \frac{A_1}{x-1} + \frac{A_2}{(x-1)^2} + \frac{A_3}{x+1} \\
        &= A_1(x-1)() \\
        4x &= (A_1+A_3)x^2 + (A_2-2A_3)x + (-A_1+A_2 +A_3) \\
        A_1 = 1 \\
        A_2 = 2 \\
        A_3 = -1
    \end{align*}

    \begin{align*}
        \int x + 1 + \frac{1}{x-1} + \frac{2}{(x-1)^2} - \frac{1}{x+1} \ dx \\
        = \frac{x^2}{2} + x + \ln|x-1| -\frac{2}{x-1} - \ln|x+1| + c
    \end{align*}

    \subsection*{Example 4}
    \begin{align*}
        \int \frac{-2x^2+4x+2}{(x-1)(x^2+3)} \ dx
    \end{align*}
    \begin{align*}
        \frac{A_1}{x-1} + \frac{A_2x+B_2}{(x^2+3)}
        &= \frac{A_1(x^2+3) + (A_2x+B_2)(x-1)}{(x-1)(x^2+3)} \\
        -2x^2 + 4x+2 &= A_1x^2 + 3A_1 + A_2x^2 + B_2x -A_2x -1 \\
        A_1 &= 1 \\
        A_2 &= -3 \\
        B_2 &= 1
    \end{align*}

    \begin{align*}
        \int \frac{1}{x-1} + \frac{-3x+1}{x^2+3} \ dx \\
        = \ln|x-1| + \int \frac{-3x}{x^2+3} + \frac{1}{x^2+3} \ dx \\
        = \ln|x-1| + \int \frac{-3x}{x^2+3} + \frac{1}{x^2+3} \ dx \\
        = \ln|x-1| - \frac{3}{2} \int \frac{1}{u} \ du + \int \frac{1}{x^2+3} \ dx \\
        = \ln|x-1| - \frac{3}{2} \ln|x^2+3| + \int \frac{1}{x^2+3} \ dx \\
        = \ln|x-1| - \frac{3}{2} \ln|x^2+3| + 1/3 \int \frac{1}{(\frac{x}{\sqrt{3}})^2+1} \ dx \\
        = \ln|x-1| - \frac{3}{2} \ln|x^2+3| + \sqrt{3}/3 \int \frac{1}{u^2+1} \ dx \\
        = \ln|x-1| - \frac{3}{2} \ln|x^2+3| + \frac{\sqrt{3}}{3} \tan^{-1}(\frac{x}{\sqrt{3}})
    \end{align*}

\end{document}
